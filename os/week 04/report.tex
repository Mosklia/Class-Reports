\documentclass{ctexart}
\usepackage{minted}
\usepackage[defaultmono]{droidsansmono}
\usepackage[hmargin=1.25in,vmargin=1in]{geometry}
\usepackage[colorlinks=true,xetex]{hyperref}

\title{实验一~Linux 下编写程序}
\author{71120226 陈宇轩}
\date{\today}

\begin{document}
    \maketitle

    \section{实验内容}

    在 Linux 下编写一个 C/C++ 程序,拷贝一个文件。

    \section{实验目的}

    \begin{itemize}
        \item 练习在 Linux 下编写并调试 C/C++ 代码,为以后其他实验做准备;
        \item 初步学习各系统调用的使用方法。 
    \end{itemize}

    \section{实验过程}

    \paragraph{安装 Linux 发行版} 使用发行版为 ArchLinux,安装镜像为 \href{https://mirrors.tuna.tsinghua.edu.cn/archlinux/iso/latest/archlinux-2022.04.05-x86_64.iso}{archlinux-2022.04.05-x86\_64.iso},具体安装过程略;

    \paragraph{安装开发相关软件包} 使用以下命令进行安装:

    \begin{minted}[autogobble]{zsh}
        sudo pacman -S gcc gdb vim
    \end{minted}

    \paragraph{编辑源码,编译}
    
    \begin{enumerate}
        \item 在合适的地方使用 \mintinline{zsh}{vim my-ls.c} 编辑源码;
        \item 使用 \mintinline{zsh}{gcc my-ls.c -o my-ls} 编译源码得到可执行文件 \mintinline{zsh}{my-ls};
        \item 使用 \mintinline{zsh}{./my-ls} 可以直接运行编译出的程序;
        \item 使用 \mintinline{zsh}{man 2 <syscall>} 可以查看系统调用\verb|syscall|的文档(特别地,使用 \verb|man 2 syscalls| 可以查看全部系统调用的列表)。
    \end{enumerate}

    \paragraph{测试程序}

    尝试在各种情况下运行程序(比如,改变源文件的所有者,改变权限,或者是给出一个不存在的文件作为输入),观察程序的输出,以及最终拷贝得到的文件内容、权限等是否正确(使用 \verb|sha256sum| 命令来计算并比较校验和,从而确保对于二进制格式文件也能得到正确的结果)。

    \section{实验体会}

    \begin{enumerate}
        \item 大多数时候我们使用的并非直接的系统调用。我们所使用的大部分系统调用实际上是 glibc 的封装,参考 \href{https://man.archlinux.org/man/glibc.7}{glibc(7)} 和 \href{https://man.archlinux.org/man/intro.2}{intro(2)};
        \item glibc 封装的系统调用中,一般返回非负整数表示正常进行,返回 -1 表示调用未能正常进行,具体的原因则以 \verb|int| 形式写入 C 标准定义的变量 \verb|errno| 中,具体处理时需要阅读相关文档来确定具体错误类型;
        \item 封装系统调用 \verb|creat| 的参数中,第二个参数是设置新建的文件的读写权限的。假设源文件的文件状态(``stat'')为 \verb|origin_stat|,则在调用 \verb|creat| 时,将第二个参数设置为 \verb|origin_stat.st_mode & 0777| 即可拷贝源文件的权限;
        \item Linux(或者说,C) 中,大量使用指针传递结构体等复杂数据结构,使用位掩码来表示信息。
    \end{enumerate}

    \section{源代码}

    在我实际的程序中,我是通过直接在命令行运行程序时提供参数来获取文件信息的,也就是模仿命令 \verb|ls| 的做法,并且对于大多数错误仅仅是简单地输出了 \verb|errno| 的值。最后,如果拷贝的目标文件存在(即拷贝操作会覆盖掉某个已存在文件),则程序会拒绝此次操作并生成错误信息。

    \begin{minted}[autogobble,linenos=true,breaklines]{c}
        // Standard Libraries
        #include <stdio.h>
        #include <stdlib.h>
        #include <errno.h>
        // glibc Headers
        #include <sys/types.h>
        #include <sys/stat.h>
        #include <fcntl.h>
        #include <unistd.h>
        
        int main(int argc, char *argv[])
        {
            if (argc < 3)
            {
                fprintf(stderr, "Usage: %s <origin> <target>\n", argv[0]);
                exit(1);
            }
        
            int origin_desc = open(argv[1], O_RDONLY);
            struct stat origin_stat;
            int origin_err = fstat(origin_desc, &origin_stat);
        
            if (origin_err != 0)
            {
                fprintf(stderr, "ERROR: Cannot open origin file %s.\n", argv[1]);
                fprintf(stderr, "ERROR: fstat() returned value %d.\n", origin_err);
                fprintf(stderr, "ERROR: Errno set to %d.\n", errno);
                exit(2);
            }
            else if (S_ISDIR(origin_stat.st_mode))
            {
                fprintf(stderr, "ERROR: Origin file %s is a directory.\n", argv[1]);
                exit(3);
            }
        
            struct stat target_stat;
            int target_err = -1;
            int target_desc;
        
            if (stat(argv[2], &target_stat) != 0)
            {
                if (errno == ENOENT)
                {
                    target_desc = creat(argv[2], origin_stat.st_mode & 07777);
                    target_err = fstat(target_desc, &target_stat);
                }
                if (target_err != 0)
                {
                    fprintf(stderr, "ERROR: Cannot create target file %s.\n", argv[1]);
                    fprintf(stderr, "ERROR: fstat() returned value %d.\n", origin_err);
                    fprintf(stderr, "ERROR: Errno set to %d.\n", errno);
                    exit(4);
                }
            }
            else // Target file exists
            {
                fprintf(stderr, "ERROR: Target file %s exists.\n", argv[2]);
                exit(5);
                // target_desc = open(argv[2], O_WRONLY);
                // target_err = fstat(target_desc, &target_stat);
                // if (target_err != 0)
                // {
                //     fprintf(stderr, "ERROR: Cannot create target file %s.\n", argv[2]);
                //     fprintf(stderr, "ERROR: fstat() returned value %d.\n", origin_err);
                //     fprintf(stderr, "ERROR: Errno set to %d.\n", errno);
                //     exit(5);
                // }
                // else if (!S_ISDIR(target_stat.st_mode))
                // {
                //     fprintf(stderr, "ERROR: Target file %s exists and is not a directory.\n", argv[2]);
                //     exit(6);
                // }
            }
        
            size_t buf_size = sizeof(char) * 50;
            void *buf = (void*)malloc(buf_size);
            int read_cnt = 0;
        
            do
            {
                read_cnt = read(origin_desc, buf, buf_size);
                write(target_desc, buf, read_cnt);
            }
            while (read_cnt == buf_size);
        
            // printf("%d\n", S_ISDIR(origin_stat.st_mode));
        
            close(target_desc);
            close(origin_desc);
        }
    \end{minted}
\end{document}
