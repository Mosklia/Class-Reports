\documentclass{ctexart}
\usepackage{amsmath}
\usepackage[hmargin=1.25in,vmargin=1in]{geometry}

\author{陈宇轩 \and 张欣}
\title{Splay Tree and its Amortized Analysis}

\begin{document}
    \maketitle

    \section{Intro}

    Hello, everyone. Today we'd like to introduce a self-balanced binary search tree, Splay tree to you.

    Splay tree was first invented in 1985, by Daniel Sleator and Robert Tarjan. In their article \textit{Self-adjusting binary search trees},
    they observed known kinds of efficient search trees, finding that efficient search trees have various drawbacks, one of which is that, the
    data structures are all designed to reduce the worst-case time per operation. Typically, search trees need to perform not one, but a
    sequence of operations. In such situations, to reduce ``average times'', or, ``amortized time'' of operations, rather than to reduce the
    time of a single one, can be a good idea.

    So, by following the principle of ``restructure simply to improve the efficiency of future operations'', Splay tree was invented. It
    uses ``splaying'', moving a specified node to the root of the tree by performing a sequence of rotations along the original path from the
    node to the root, as its restructuring heuristics.
\end{document}
