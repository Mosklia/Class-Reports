\documentclass{ctexbeamer}
\usepackage{amsmath,amssymb}
\usepackage{amsthm}
\usepackage{subfig}
\usepackage{graphicx}
\usepackage[defaultmono,scale=0.85]{droidsansmono}
\usepackage{tikz}
\usepackage{minted}
\usepackage{color}
% \usepackage{mdframed}
% \usepackage[colorlink=true,xetex]{hyperref}

\usetheme{Madrid}
\usefonttheme[onlymath]{serif}
\definecolor{codebg}{rgb}{0.95,0.95,0.95}
\setCJKmonofont{WenQuanYi Micro Hei Mono}
% \surroundwithmdframed{minted}

\usetikzlibrary{graphdrawing, graphs, quotes, shapes}
\usegdlibrary{trees}
\tikzset{
    noleaf/.style={draw,ellipse,fill=red!60},
    ref/.style={draw,ellipse,fill=green!40},
    normal/.style={draw,ellipse},
    every picture/.style={}
}

\setminted{bgcolor=codebg,autogobble,breaklines}
\setmintedinline{bgcolor=codebg}

\newcommand{\theauthor}{Sparky\_14145}
\newcommand{\theinst}{College of Software Engineering}
\newcommand{\theshortinst}{CoSE}

\input{personal_info/info.tex}

\author{\theauthor}
\institute[\theshortinst]{\theinst}
\title{使用 Triton 实现符号执行}
\subtitle{第 13 章}

\begin{document}
    \begin{frame}
        \titlepage
    \end{frame}

    \begin{frame}
        \frametitle{目录}
        \tableofcontents
    \end{frame}

    \section{Triton 简介}
    \begin{frame}
        \frametitle{Triton 简介}
    
        Triton 是一款免费的开源二进制文件分析库,允许用户构建自己的二进制分析程序,以对符号执行的支持而闻名。

        Triton 拥有以下特性:

        \begin{enumerate}
            \item 支持动态符号执行;\pause
            \item 支持污点分析;\pause
            \item 支持以下指令集架构的 AST 表示:x86, x86\_64, ARM32 和 AArch64;\pause
            \item 提供 C++ 和 Python API;\pause
            \item 以及更多其他特性...
        \end{enumerate}
    
    \end{frame}

    \section{抽象语法树}
    \begin{frame}[fragile]
        \frametitle{抽象语法树}
    
        Triton 将符号表达式和约束条件表示为抽象语法树(abstract syntax tree, AST)。

        抽象语法树描述了操作和操作数之间的语法关系,如图 \ref{fig:ast-example} 所示为汇编指令 \mintinline{nasm}{shr eax, cl} 所对应的抽象语法树:

        \begin{figure}
            \centering
            \begin{tikzpicture}
                \graph[binary tree layout, nodes=normal]{
                    "bvlshr"[noleaf] -> {
                        bv1[as=bv,noleaf] -> {"$\alpha_1$", "32"},
                        "concat"[noleaf] -> {
                            bv2[as=bv,noleaf] -> {"0", "24"},
                            bv3[as=bv,noleaf] -> {"$\alpha_2$", "8"}
                        }
                    }
                };
            \end{tikzpicture}
            \caption{抽象语法树举例}
            \label{fig:ast-example}
        \end{figure}

    
    \end{frame}

    \subsection{使用引用的抽象语法树}

    \begin{frame}
        \frametitle{使用引用的抽象语法树}
        \framesubtitle{抽象语法树}
    
        图 \ref{fig:ast-example} 展示了一棵完整抽象语法树。可以看到,完整抽象语法树往往包含了所有结点的信息,因此无法较好地复用各个子树来节省内存空间。\pause

        一种解决方案就是使用引用。\pause

        \begin{figure}
            \small
            \centering
            \subfloat[\texttt{eax}的抽象语法树]{
                \centering
                \begin{tikzpicture}
                    \graph[binary tree layout, nodes=normal]{
                        bv["ref!1",noleaf] -> {"$\alpha_1$", 32}
                    };
                \end{tikzpicture}
            }\;
            \subfloat[\texttt{cl}的抽象语法树]{
                \centering
                \begin{tikzpicture}
                    \graph[binary tree layout, nodes=normal]{
                        bv["ref!2",noleaf] -> {"$\alpha_2$", 8}
                    };
                \end{tikzpicture}
            }\;
            \subfloat[\texttt{cl}的抽象语法树]{
                \centering
                \begin{tikzpicture}
                    \graph[binary tree layout, nodes=normal]{
                        "bvlshr"[noleaf] -> {
                            "ref!1"[ref],
                            "concat"[noleaf] -> {
                                bv2[as=bv,noleaf] -> {"0", "24"},
                                "ref!2"[ref]
                            }
                        }
                    };
                \end{tikzpicture}
            }\\
            \caption{使用引用重构的抽象语法树}
            \label{fig:ast-with-ref}
        \end{figure}
    
    \end{frame}

    \begin{frame}
    
        如 \ref{fig:ast-with-ref} 所示,通过设立一种特殊的``引用结点'',Triton 在构建大型的抽象语法树时,无需完整地复制所有的结点,而只需要使用引用结点就能引用别的语法树,从而提高了性能。
    
    \end{frame}

    \section{使用 Triton 进行后向切片}

    \begin{frame}[fragile]
        \frametitle{使用 Triton 进行后向切片}
    
        后向切片是一种二进制分析技术,它可以告诉你在执行过程中的某个时刻,之前的哪些指令会影响给定寄存器或内存地址的值。\pause

        举例来说有一个未知的可执行文件 \mintinline{text}{test},如清单所示:

        {
            \scriptsize
            \begin{minted}[highlightlines={8}]{text}
                $ objdump -M intel -d test
                ...
                4010cc:	00 00 00 
                4010cf:	90                   	nop
                4010d0:	b8 40 40 40 00       	mov    eax,0x404040
                4010d5:	48 3d 40 40 40 00    	cmp    rax,0x404040
                4010db:	74 13                	je     4010f0 <_ZNSolsEi@plt+0x70>
                4010dd:	48 8b 05 e4 2e 00 00 	mov    rax,QWORD PTR [rip+0x2ee4]
                4010e4:	48 85 c0             	test   rax,rax
                ...
            \end{minted}
        }
    
    \end{frame}

    \begin{frame}[fragile]

        我们想要知道在地址 \mintinline{text}{4010dd} 及之前哪些指令影响过寄存器 \mintinline{text}{rax} 的值,观察可知其中应至少包含以下指令:

        {
            \scriptsize
            \begin{minted}{text}
                4010d0:	b8 40 40 40 00       	mov    eax,0x404040
                4010dd:	48 8b 05 e4 2e 00 00 	mov    rax,QWORD PTR [rip+0x2ee4]
            \end{minted}
        }

        于是我们尝试用 Triton 实现。\pause

        整体实现的伪代码如下:

        {
            \small
            \begin{minted}{cpp}
                void backward_slicing() {
                    打开二进制文件;
                    设定体系结构;
                    根据配置文件初始化内存与寄存器;
                    逐条分析指令;
                }
            \end{minted}
        }
    
    \end{frame}

    \begin{frame}[fragile]
        \frametitle{打开二进制文件}
        \framesubtitle{使用 Triton 进行后向切片}
    
        使用第四章使用的二进制加载器,打开二进制文件的代码很简单:

        {
            \small
            \begin{minted}{cpp}
                std::string fname(argv[1]);
                // if(load_binary(fname, &bin, Binary::BinaryFormat::AUTO) < 0) return 1;
                if (Binary *tmp = Binary::from_file(fname); tmp == nullptr) {
                    return 1;
                } else {
                    bin = *tmp;
                }
            \end{minted}
        }
    \end{frame}

    \begin{frame}[fragile]
        \frametitle{设定体系结构}
        \framesubtitle{使用 Triton 进行后向切片}
    
        设置体系结构的过程由一个名为 \mintinline{cpp}{set_triton_arch} 的函数完成:

        {
            \footnotesize
            \begin{minted}{cpp}
                if(set_triton_arch(bin, api, ip) < 0) return 1;
            \end{minted}
        }
        \pause
        而这个函数也只是检查文件的架构是否为 x86,并依据位宽调整 Triton API。核心代码如下:

        {
            \footnotesize
            \begin{minted}{cpp}
                int set_triton_arch(bin, api, ip) {
                    // ...(检查是否为 X86 架构)
                    if (bin.bits == 32) {        // 32 位
                        api.setArchitecture(triton::arch::ARCH_X86);
                        ip = triton::arch::ID_REG_X86_EIP;
                    } else if (bin.bits == 64) { // 64 位
                        api.setArchitecture(triton::arch::ARCH_X86_64);
                        ip = triton::arch::ID_REG_X86_RIP;
                    }
                    // ...(提示错误,并根据是否成功设置返回不同的返回值)
                }
            \end{minted}
        }
    
    \end{frame}

    \begin{frame}[fragile]
        \frametitle{根据配置文件初始化内存与寄存器}
        \framesubtitle{使用 Triton 进行后向切片}
    
        这一部分主要调用 Triton API 完成,具体而言就是遍历配置文件中每一处内存与寄存器,并将其初始值设定为配置文件中的给定值;而解析配置文件的操作也可以由 \mintinline{text}{triton_util} 内的函数完成:

        {
            \small
            \begin{minted}{cpp}
                std::map<triton::arch::register_e, uint64_t> regs;
                std::map<uint64_t, uint8_t> mem;

                if(parse_sym_config(argv[2], &regs, &mem) < 0) return 1;
                for(auto &kv: regs) {   // 处理寄存器
                    triton::arch::Register r = api.getRegister(kv.first);
                    api.setConcreteRegisterValue(r, kv.second);
                }
                for(auto &kv: mem) {    // 处理内存
                    api.setConcreteMemoryValue(kv.first, kv.second);
                }
            \end{minted}
        }
    
    \end{frame}

    \begin{frame}[fragile]
        \frametitle{逐条分析指令}
        \framesubtitle{使用 Triton 进行后向切片}
    
        逐条分析指令是后向切片程序的重头戏,这部分的伪代码如下:

        {
            \small
            \begin{minted}{cpp}
                将 PC 初始化为给定入口点;
                将 slice_addr 设定为给定切片地址;
                while (PC 位于 .text 段中) {
                    解析并反汇编 PC 所指向的一条命令;
                    将指令的人类可读形式保存为指令的注释;
                    if (PC == slice_addr) {
                        输出切片;
                        break;
                    }
                    更新 PC;
                }
            \end{minted}
        }
    
    \end{frame}

    \begin{frame}[fragile]
    
        下面来看具体的代码:

        {
            \small
            \begin{minted}[highlightlines={1-2}]{cpp}
                uint64_t pc         = strtoul(argv[3], NULL, 0);
                uint64_t slice_addr = strtoul(argv[4], NULL, 0);
                Section *sec = bin.get_text_section();
                
                while (sec->contains(pc)) {
                    char mnemonic[32], operands[200];
                    int len = disasm_one(sec, pc, mnemonic, operands);
                    if(len <= 0) return 1;

                    triton::arch::Instruction insn;
                    insn.setOpcode(&(sec->bytes[pc-sec->vma]), len);
                    insn.setAddress(pc);

                    api.processing(insn);
                    // ...
                }
            \end{minted}
        }

        直接从命令行参数取得切片入口和切片地址。
    
    \end{frame}

    \begin{frame}[fragile]

        {
            \small
            \begin{minted}[highlightlines={3-5}]{cpp}
                uint64_t pc         = strtoul(argv[3], NULL, 0);
                uint64_t slice_addr = strtoul(argv[4], NULL, 0);
                Section *sec = bin.get_text_section();
                
                while (sec->contains(pc)) {
                    char mnemonic[32], operands[200];
                    int len = disasm_one(sec, pc, mnemonic, operands);
                    if(len <= 0) return 1;

                    triton::arch::Instruction insn;
                    insn.setOpcode(&(sec->bytes[pc-sec->vma]), len);
                    insn.setAddress(pc);

                    api.processing(insn);
                    // ...
                }
            \end{minted}
        }

        循环条件:PC 必须在文件的 \mintinline{text}{.text} 节中。\pause

        思考:这样的条件可能带来什么样的缺陷?
    
    \end{frame}

    \begin{frame}[fragile]

        {
            \small
            \begin{minted}[highlightlines={6-8}]{cpp}
                uint64_t pc         = strtoul(argv[3], NULL, 0);
                uint64_t slice_addr = strtoul(argv[4], NULL, 0);
                Section *sec = bin.get_text_section();
                
                while (sec->contains(pc)) {
                    char mnemonic[32], operands[200];
                    int len = disasm_one(sec, pc, mnemonic, operands);
                    if(len <= 0) return 1;

                    triton::arch::Instruction insn;
                    insn.setOpcode(&(sec->bytes[pc-sec->vma]), len);
                    insn.setAddress(pc);

                    api.processing(insn);
                    // ...
                }
            \end{minted}
        }

        使用 \mintinline{text}{capstone} 反汇编当前指令。\pause

        其中 \mintinline{cpp}{disasm_one()} 是定义在 \mintinline{text}{disasm_util.cc} 中的函数,这里不展开讲。
    
    \end{frame}

    \begin{frame}[fragile]

        {
            \small
            \begin{minted}[highlightlines={10-14}]{cpp}
                uint64_t pc         = strtoul(argv[3], NULL, 0);
                uint64_t slice_addr = strtoul(argv[4], NULL, 0);
                Section *sec = bin.get_text_section();
                
                while (sec->contains(pc)) {
                    char mnemonic[32], operands[200];
                    int len = disasm_one(sec, pc, mnemonic, operands);
                    if(len <= 0) return 1;

                    triton::arch::Instruction insn;
                    insn.setOpcode(&(sec->bytes[pc-sec->vma]), len);
                    insn.setAddress(pc);

                    api.processing(insn);
                    // ...
                }
            \end{minted}
        }

        复制指令,并让 Triton 去模拟执行这条指令。
    
    \end{frame}

    \begin{frame}[fragile]

        {
            \scriptsize
            \begin{minted}[highlightlines={3-6}]{cpp}
                while (sec->contains(pc)) {
                    // ...
                    for (auto &se: insn.symbolicExpressions) {
                        std::string comment = mnemonic; comment += " "; comment += operands;
                        se->setComment(comment);
                    }

                    if (pc == slice_addr) {
                        print_slice(api, sec, slice_addr, get_triton_regnum(argv[5]), argv[5]);
                        break;
                    }

                    pc = (uint64_t)api.getConcreteRegisterValue(api.getRegister(ip));
                }
            \end{minted}
        }

        为每条符号化的指令加上注释。\pause

        因为 Triton 在内部会将指令转化为 SMT-LIB 2.0 格式的符号表达式,所以一条汇编指令可能会产生多个符号表达式。为了从这种表达式生成 x86 汇编指令,我们需要遍历所有的符号表达式。
    
    \end{frame}

    \begin{frame}[fragile]

        {
            \scriptsize
            \begin{minted}[highlightlines={8-11}]{cpp}
                while (sec->contains(pc)) {
                    // ...
                    for (auto &se: insn.symbolicExpressions) {
                        std::string comment = mnemonic; comment += " "; comment += operands;
                        se->setComment(comment);
                    }

                    if (pc == slice_addr) {
                        print_slice(api, sec, slice_addr, get_triton_regnum(argv[5]), argv[5]);
                        break;
                    }

                    pc = (uint64_t)api.getConcreteRegisterValue(api.getRegister(ip));
                }
            \end{minted}
        }

        打印给定的寄存器的切片。\pause

        \mintinline{cpp}{get_triton_regnum} 是定义于 \mintinline{text}{triton_util.cc} 的一个函数,用来将字符串转化为 \mintinline{cpp}{triton::register_e} 枚举类型,实现较简单。
    
    \end{frame}

    \begin{frame}[fragile]

        {
            \scriptsize
            \begin{minted}[highlightlines={8-11}]{cpp}
                while (sec->contains(pc)) {
                    // ...
                    for (auto &se: insn.symbolicExpressions) {
                        std::string comment = mnemonic; comment += " "; comment += operands;
                        se->setComment(comment);
                    }

                    if (pc == slice_addr) {
                        print_slice(api, sec, slice_addr, get_triton_regnum(argv[5]), argv[5]);
                        break;
                    }

                    pc = (uint64_t)api.getConcreteRegisterValue(api.getRegister(ip));
                }
            \end{minted}
        }

        更新 PC 的值,使其指向下一条指令(Triton 内部寄存器 \mintinline{text}{ip} 已经被自动更新了)。
    
    \end{frame}

    \begin{frame}[fragile]
        \frametitle{计算后向切片}
        \framesubtitle{使用 Triton 进行后向切片}
    
        计算后向切片的实质就是输出所有 Triton 执行过的,与目标寄存器有关的指令。\pause

        代码中计算后向切片的函数为 \mintinline{text}{print_slice},其流程如下:\pause

        \begin{enumerate}
            \item 获取目标寄存器的符号表达式 \mintinline{text}{regExpr};\pause
            \item 获取符号表达式 \mintinline{text}{regExpr} 的切片;\pause
            \item 输出切片中每一个符号表达式的注释;\pause
            \item 反汇编当前(切片地址)的一条指令。\pause
        \end{enumerate}

        核心代码如下:

        {
            \small
            \begin{minted}{cpp}
                auto regExpr = api.getSymbolicRegisters()[reg];
                auto slice = api.sliceExpressions(regExpr);
                for (auto &kv: slice) {
                    printf("%s\n", kv.second->getComment().c_str());
                }
                // 反编译一条指令的代码可参考 main 函数
            \end{minted}
        }
    
    \end{frame}

    \section{使用 Triton 提升代码覆盖率}
    \begin{frame}[fragile]
        \frametitle{使用 Triton 提升代码覆盖率}

        提升代码覆盖率的目的在于:探索程序运行时,已知某个输入对应的输出时,希望得到另一组输入,且此输入会产生另一个输出。
    
        \mintinline{cpp}{code_coverage} 的伪代码如下:
    
        {
            \small
            \begin{minted}{cpp}
                void backward_slicing() {
                    打开二进制文件;
                    设定体系结构;
                    根据配置文件初始化内存与寄存器;
                    符号化内存与寄存器;
                    逐条分析指令;
                }
            \end{minted}
        }

        其中打开二进制文件、设定体系结构与初始化过程与后向切片相同,不再赘述。
    \end{frame}

    \begin{frame}[fragile]
        \frametitle{符号化内存与寄存器}
        \framesubtitle{使用 Triton 提升代码覆盖率}

        实际上只有那些配置文件调整过的内存与寄存器需要手动符号化。
    
        符号化内存与寄存器的代码如下:

        {
            \small
            \begin{minted}{cpp}
                for(auto regid: symregs) {
                    triton::arch::Register r = api.getRegister(regid);
                    api.symbolizeRegister(r)->setComment(r.getName());
                }
                for(auto memaddr: symmem) {
                    api.symbolizeMemory(triton::arch::MemoryAccess(memaddr, 1))
                        ->setComment(std::to_string(memaddr));
                }
            \end{minted}
        }\pause

        主要工作就是利用 \mintinline{cpp}{symbolize} 系列方法和 \mintinline{cpp}{setComment} 来设置注释方便以后打印。
    
    \end{frame}

    \begin{frame}[fragile]
        \frametitle{逐条分析指令}
        \framesubtitle{使用 Triton 提升代码覆盖率}
    
        逐条分析指令的伪代码如下:
        
        {
            \small
            \begin{minted}{cpp}
                将 PC 初始化为给定入口点;
                将 branch_addr 设定为给定切片地址;
                while (PC 位于 .text 段中) {
                    解析并反汇编 PC 所指向的一条命令;
                    将指令的人类可读形式保存为指令的注释;
                    if (PC == branch_addr) {
                        寻找新分支路径;
                        break;
                    }
                    更新 PC;
                }
            \end{minted}
        }

        实际上,除了寻找新分支路径以外所有的部分都与后向切片相同,故不赘述。
    
    \end{frame}

    \begin{frame}[fragile]
        \frametitle{寻找新分支路径}
        \framesubtitle{使用 Triton 提升代码覆盖率}
    
        寻找新分支路径的伪代码如下:

        {
            \small
            \begin{minted}{cpp}
                void find_new_input() {
                    初始化约束列表为 true == true;
                    for (pc : 所有带分支的路径约束条件) {
                        for (bc : pc 上的所有分支约束条件) {
                            if (此分支不是我们想反转的分支 && 此分支被选中) {
                                将此分支的条件以逻辑 AND 方式添加到约束中;
                            } else if (此分支未被选中) {
                                选中此分支;
                                将此分支的约束以逻辑 AND 方式添加到约束中;
                                尝试求出并输出约束的一组解;
                            }
                        }
                    }
                }
            \end{minted}
        }
    
    \end{frame}

    \begin{frame}[fragile]
    
        初始化约束列表为 \mintinline{cpp}{true == true}:

        {
            \small
            \begin{minted}{cpp}
                auto ast = api.getAstContext();
                auto constraint_list = ast->equal(ast->bvtrue(), ast->bvtrue());
            \end{minted}
        }
        \pause 程序的所有路径约束可以用 \mintinline{cpp}{API::getPathConstraints()} 来获取,且某个路径的分支约束可以用 \mintinline{cpp}{PathConstraint::getBranchConstraints()} 来获取,于是得到两层循环头的代码:

        {
            \small
            \begin{minted}{cpp}
                const std::vector<triton::engines::symbolic::PathConstraint>
                    &path_constraints = api.getPathConstraints();
                for (auto &pc : path_constraints) {
                    if (!pc.isMultipleBranches())
                        continue;
                    for (auto &branch_constraint : pc.getBranchConstraints()) {
                        // ...(处理单条分支约束)
                    }
                }
            \end{minted}
        }
    
    \end{frame}

    \begin{frame}[fragile]
    
        而处理分支约束的代码如下所示:

        {
            \scriptsize
            \begin{minted}{cpp}
                bool flag = std::get<0>(branch_constraint);
                uint64_t src_addr = std::get<1>(branch_constraint);
                uint64_t dst_addr = std::get<2>(branch_constraint);
                auto constraint = std::get<3>(branch_constraint);

                if (src_addr != branch_addr) {
                    if (flag) {
                        constraint_list = ast->land(constraint_list, constraint);
                    }
                } else {
                    printf("    0x%jx -> 0x%jx (%staken)\n", src_addr, dst_addr, flag ? "" : "not ");
                    if (!flag) {
                        printf("    computing new input for 0x%jx -> 0x%jx\n", src_addr, dst_addr);
                    constraint_list = ast->land(constraint_list, constraint);
                    for (auto &kv : api.getModel(constraint_list)) {
                        printf("      SymVar %lu (%s) = 0x%jx\n", kv.first,
                            api.getSymbolicVariable(kv.first)->getComment().c_str(),
                                (uint64_t)kv.second.getValue());
                        }
                    }
                }
            \end{minted}
        }
    
    \end{frame}

    \begin{frame}[fragile]
        \frametitle{测试代码覆盖工具}
        \framesubtitle{使用 Triton 提升代码覆盖率}
    
        编译 \mintinline{text}{branch.c},然后反编译:

        {
            \tiny
            \begin{minted}[highlightlines={6-7}]{objdump-nasm}
              ...
              0000000000401146 <branch>:
                401146:	55                   	push   rbp
                401147:	48 89 e5             	mov    rbp,rsp
                40114a:	48 83 ec 10          	sub    rsp,0x10
                40114e:	89 7d fc             	mov    DWORD PTR [rbp-0x4],edi
                401151:	89 75 f8             	mov    DWORD PTR [rbp-0x8],esi
                401154:	83 7d fc 04          	cmp    DWORD PTR [rbp-0x4],0x4
                401158:	7f 28                	jg     401182 <branch+0x3c>
                40115a:	83 7d f8 0a          	cmp    DWORD PTR [rbp-0x8],0xa
                40115e:	75 11                	jne    401171 <branch+0x2b>
                401160:	48 8d 05 9d 0e 00 00 	lea    rax,[rip+0xe9d]        # 402004 <_IO_stdin_used+0x4>
                401167:	48 89 c7             	mov    rdi,rax
                40116a:	e8 c1 fe ff ff       	call   401030 <puts@plt>
                40116f:	eb 20                	jmp    401191 <branch+0x4b>
                401171:	48 8d 05 9d 0e 00 00 	lea    rax,[rip+0xe9d]        # 402015 <_IO_stdin_used+0x15>
                401178:	48 89 c7             	mov    rdi,rax
                40117b:	e8 b0 fe ff ff       	call   401030 <puts@plt>
                401180:	eb 0f                	jmp    401191 <branch+0x4b>
                401182:	48 8d 05 9d 0e 00 00 	lea    rax,[rip+0xe9d]        # 402026 <_IO_stdin_used+0x26>
                401189:	48 89 c7             	mov    rdi,rax
                40118c:	e8 9f fe ff ff       	call   401030 <puts@plt>
                401191:	90                   	nop
                401192:	c9                   	leave
                401193:	c3                   	ret
              ...
            \end{minted}
        }
    
    \end{frame}
\end{document}