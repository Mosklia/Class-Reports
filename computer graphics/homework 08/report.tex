\documentclass{ctexart}

\usepackage{graphicx}
\usepackage[defaultmono,scale=0.85]{droidsansmono}
\usepackage{amsmath}
\usepackage{amssymb}
\usepackage[hmargin=1.1in,vmargin=1in]{geometry}

\newcommand{\theauthor}{Sparky\_14145}

\input{personal_info/info.tex}

\title{计算机图形 作业六}
\author{\theauthor}

\begin{document}
    \maketitle

    \section{函数流程}

    \subsection{\texttt{Rope::Rope}}

    整条绳索的构造函数。

    具体的操作流程如下:
    \begin{enumerate}
        \item 对于每个绳子结点:
        \begin{enumerate}
            \item 计算出它所在的位置(公式:$\mathrm{pos} = t \times \mathrm{end} + (1 - t) \times \mathrm{begin}$,其中 $t$ 结点编号减 1 除以结点总数减 1);
            \item 利用其位置、质量创建质点,将其速度设为 0,并且将其加入质点集合;
            \item 用线段(\texttt{String})将其与前一个质点相连接。
        \end{enumerate}
        \item 将固定结点列表中的点标记为固定。
    \end{enumerate}

    \subsection{\texttt{Rope::simulateEuler}}

    模拟半隐式欧拉法。

    首先计算每条线对每个质点的作用力,对各个质点进行累加。然后将重力再加到质点上,用 $F = ma$、$\Delta v = a \Delta t$ 与 $\Delta x = v \Delta t$ 计算出质点的新速度与新位置,最后清空累加的力(便于下一轮模拟计算)。

    \subsection{\texttt{Rope::simulateVerlet}}

    先同半隐式欧拉法计算出加速度,然后套公式更新坐标就行了。

    \section{运行结果}

    \begin{center}
        \includegraphics[width=0.9\textwidth]{pics/result-1.png} \\
        \includegraphics[width=0.9\textwidth]{pics/result-2.png}
    \end{center}
\end{document}