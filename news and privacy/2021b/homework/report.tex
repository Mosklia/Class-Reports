\documentclass{ctexart}
\usepackage[hmargin=1in,vmargin=1in]{geometry}
\usepackage[colorlinks=true,bookmarks=true,bookmarksnumbered=true]{hyperref}

\author{71120226 陈宇轩}
\title{数字经济发展潮流不可挡}
 
\bibliographystyle{IEEEtran}

\begin{document}
    \maketitle

    \section{总述}

    2021 年上半年,中国数字经济持续快速平稳增长。2021 年 1 至 5 月,计算机、通信和其他电子设备
    制造业增加值同比增长 21.5\%,两年复合增长率明显高于全部工业的复合增长率。上半年实物商品网上
    零售额两年平均增长 16.5\%,占社会消费品零售总额的比重达到了 23.7\%,7月全国快递业务量已接
    近2018年全年水平。跨境电商仍保持高位增长,已成为稳外贸的重要力量;农村电商提质升级,为乡村振兴
    注入活力。数字基础设施加速普及,固定宽带普及率已达到发达国家水平。
    \cite{2021上半年数字经济总体形势分析}

    \section{原因分析}
    \subsection{新冠疫情的爆发助推电商行业快速增长}

    2020 年,新冠疫情爆发以来,至今各地仍偶有确诊的情况出现。数字经济,尤其是电商,相比传统线下
    经济,在安全,便捷等方面有着不可比拟的优势,快递配送更是有效减少了人员流动与聚集,为抗疫作出
    了极大的贡献。疫情使得不少家庭逐渐认识到数字经济的重要与便捷,并且使得他们乐于参与其中。

    \subsection{政策利好数字经济行业增长}

    2015 年,国务院通过了《促进大数据行动纲要》,《纲要》极富远见地规划了我国大数据发展和应用的
    目标与计划,强调了大数据发展与相关政策的衔接配合,指出了近几年内有待“明确”的事项
    \cite{《促进大数据发展行动纲要》解读}。在《纲要》的指引下,福建、重庆、浙江、广东等各省市
    出台了一系列促进数字化社会建设与数字经济行业增长的、明确的、切实可行的政策,努力让社会经济
    转型、升级\cite{深化数字化转型引领高质量发展},这使得数字经济行业高速发展。

    \subsection{“大数据”热提升了全民的数据意识}

    长期以来,中国人习惯于定性思维而不是定量思维,这阻碍了现代科技在中国的发展。而近几年的大数据热,
    强化了社会的数据意识,让社会,让群众逐渐接受了定量思维,一定程度上提升了社会的科学素养,让社会
    学习理性思维的科学精神\cite{《促进大数据发展行动纲要》解读},为数字经济发展作了铺垫,创造了机会。

    \subsection{大数据和数字化是社会发展的大势所趋}
 
    大数据是我国信息化发展步入深水区后的核心主题和战略抉择。随着近年来,我国的信息化水平不断提高,
    而旧式的“各扫门前雪”的信息系统和管理模式已经难以解决经济社会发展的难题。\cite{《促进大数据发展行动纲要》解读}
    为此,政府各部门的信息与信息化管理必须利用好大数据和数字化,而经济领域也不例外。可以说,大数据和
    数字化不仅是社会发展与管理的大势所趋,更是经济发展的必然路径。

    \subsection{近年来我国科技与经济高速发展与进步创造了有利条件}

    近年来,PC,互联网乃至智能手机的不断普及与升级,让大众有了接触,参与数字经济的机会。互联网的普及
    推动了网购,在线交易的增长,而经济的发展则让百姓更容易购买电子产品,在向大众提供更多线上交易机会
    的同时促进了电子制造业的发展。2021年1-5月份,计算机、通信和其他电子设备制造业增加值同比增长21.5\%,
    实物商品网上零售额两年平均增长16.5\%,占社会消费品零售总额的比重达到了23.7\%
    \cite{2021上半年数字经济总体形势分析}。同时人工智能,5G 通信等技术的迅猛发展,更是为数字经济的发展
    注入了活力。2020年,最复杂的自然语言深度学习模型参数量首次突破了千亿大关,达到了1750亿。按照当前趋势
    预测,到2023年,这类模型的参数量有望达到人脑神经突触数量,约125万亿
    \cite{单志广:智算中心是构建智慧社会和智能经济的关键性公共算力基础设施},这必将令大数据与数字经济
    的应用变得更加广泛。

    \bibliography{citations}
\end{document}