\documentclass[12pt]{ctexrep}
\usepackage[hmargin=1.25in,vmargin=1in]{geometry}
\usepackage{gbt7714}
\usepackage[colorlinks=true,xetex]{hyperref}

\linespread{1.1}
\bibliographystyle{gbt7714-numerical}

\author{71120226~陈宇轩}
\title{形势与政策~研讨报告}

\begin{document}
    \maketitle

    \chapter{全面从严治党}

    全面从严治党,首次提出于 1987 年 10 月的十三大报告,并于 1992 年 10 月的十四大首次写入总纲 \cite{cit1.1},主要包含六个“从严”:
    \begin{enumerate}
        \item 抓思想从严:坚持用马克思主义中国化最新成果武装头脑、凝心聚魂,用理想信念和党性教育固本培元、补钙壮骨,着力教育引导全党坚定理想、坚定信念,增强中国特色社会主义道路自信、理论自信、制度自信、文化自信;
        \item 抓管党从严:坚持和落实党的领导,引导全党增强政治意识、大局意识、核心意识、看齐意识,着力落实管党治党责任,不断增强各级党组织管党治党意识和能力;
        \item 抓执纪从严:坚持把纪律挺在前面,严明党的政治纪律和政治规矩,坚持有令必行、有禁必止,坚决查处各种违反纪律的行为,使各项纪律规矩真正成为“带电的高压线”,用铁的纪律从严治党,保证全党团结统一、步调一致;
        \item 抓治吏从严:坚持正确用人导向,深化干部人事制度改革,破解“四唯”难题,着力整治用人上的不正之风,优化选人用人环境;
        \item 抓作风从严:从落实八项规定和整治“四风”入手,坚持以上率下,锲而不舍、扭住不放,着力解决问题,推动党风政风好转;
        \item 抓反腐从严:坚持以零容忍态度惩治腐败,“老虎”“苍蝇”一起打,着力扎紧制度的笼子,遏制腐败蔓延势头。
    \end{enumerate}

    而推进全面从严治党,有以下途径:\cite{cit1.2}
    \begin{enumerate}
        \item 解决好思想理论建设同时要不断加强思想道德建设;
        \item 制定科学的制度体系,完善畅通执行渠道;
        \item 畅通群众监督的渠道和途径,使监督落到实处;
        \item 不断认识把握党的建设规律;
        \item 把抓好党建作为最大政绩,认真探索党建工作对经济和其他工作作用的途径和方法。
    \end{enumerate}

    \chapter{我国经济社会发展}

    2021 年是“十四五”规划和全面建设社会主义现代化国家新征程开局之年,我国经济恢复取得明显成效,主要有以下十个亮点\cite{cit2.1}:
    \begin{enumerate}
        \item 经济增长国际领先。前三季中,我国的经济增速明显大于世界主要经济体,并显现出发展韧性强、潜力大、动力足的特点;
        \item 城镇新增就业持续扩大。10月时即完成全年新增就业人口目标,大学生与农民工的失业率均有所下降,体现出就业帮扶精准有效;
        \item 居民消费价格处于合理区间。市场保供稳价力度不断扩大,CPI保持温和上涨态势,食品价格稳中略降;
        \item 对外贸易和利用外资较快增长。对外贸易量增质升,结构继续优化,进出口总额创历史同期新高;
        \item 粮食连年增产丰收。夏粮面积、单产、总产“三增加”,早稻连续两年增产,全年总粮食产量再创历史新高;
        \item 制造业生产投资稳定发展。制造强国战略深入实施,先进制造业和现代服务业融合发展得到加强,制造业生产较快增长,带动投资持续恢复,投资增长态势稳定;
        \item 新产业新业态茁壮成长。创新驱动发展战略有效实施,新一代信息技术加速向网络购物、移动支付、线上线下融合等新型消费领域渗透融合,新业态新模式持续活跃;
        \item 居民收入增长与经济增长基本同步。前三季度,全国居民人均可支配收入增长,且与经济增长保持同步,城乡居民收入差距进一步缩小;
        \item 生态环境保护明显加强。前三季度清洁能源消费量占能源消费总量的比重提高,全国多地PM2.5平均浓度下降,地表水断面水质优良比例提高;
        \item 对世界经济的贡献提升。我国经济总量占世界经济比重连年稳步提高,大量向国际社会提供新冠疫苗,货物进口额占全球的比重创历史新高。
    \end{enumerate}

    \chapter{港澳台工作}

    台湾自古以来一直是中国的不可或缺的一部分,“一个中国”原则更是我国与其他国家正常交往的前提条件之一。然而国际上总有些政治团体不安好心,公然挑衅我国主权,制造了不限于以下事件:
    \begin{itemize}
        \item 美国国会议员窜访台湾。4 月 14 日晚,美国 6 名反华参众议员晚摸黑抵达台北,并于次日与蔡英文会面;
        \item 蔡英文多次表示积极争取参与“印太经济框架”。“印太经济框架”是美国基于自身利益,企图“去中国化”而提出的经济框架,蔡英文曾多次表示会努力加入,然而目前仍未收到美国的邀请;
        \item 美国官员将台湾和乌克兰问题相提并论。4月6日,美国财政部长耶伦表示:如中国大陆“侵略”台湾,美有能力和决心采取与制裁俄侵乌同样的行动。同日,美国常务副国务卿舍曼表示,中方应从西方在俄乌冲突问题上的协调反应中吸取教训,即任何武力夺取台湾的行为均不可接受;
        \item 美国向台湾出售武器。4月5日,美国防安全合作局发布消息称,美国务院已批准向“驻美台北经文处”出售总额9500万美元的军事技术及相关设备,主要包括为爱国者防空系统提供相关训练、计划、部署、运行、维修、支援和其他有关设备等;
        \item 美国邀请台湾参加所谓“领导人民主峰会”。2021 年 12 月 9 日至 10 日,美国召开“领导人民主峰会”,邀请台湾参加,却没有邀请中国和俄罗斯。
    \end{itemize}
    
    我们希望,这些政治团体能够认识到“一个中国”的意义与必要性,改正自身的认知错误,进而与中国友好交流。

    \chapter{国际形势与政策}

    2 月 24 日,俄罗斯在乌克兰顿巴斯地区展开特别军事行动,其正式标志着俄乌战争的爆发。

    此后俄罗斯遭到以美国为首的西方多国的经济等一系列制裁\cite{cit4.1}。俄罗斯公布了一个包含美国,欧盟成员国,英国,乌克兰,日本等国家的“不友好国家和地区”名单,并在向其供应天然气时采用卢布进行结算\cite{cit4.2}。

    乌克兰则尝试加入欧盟组织,或是请求北约等组织援助自身或参与制裁,对抗俄罗斯。

    由于俄罗斯是石油出口大国,且俄乌均为粮食出口大国,此冲突对欧洲多国经济产生了重大负面影响。

    与此同时,巴勒斯坦民众与以色列警察在位于耶路撒冷老城的阿克萨清真寺发生冲突,造成160多名巴勒斯坦人受伤,另有约400人被以色列警方拘捕\cite{cit4.3}。
    
    \bibliography{citations}
\end{document}