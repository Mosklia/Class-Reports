\documentclass[12pt,a4paper]{ctexart}
\usepackage{geometry}
\geometry{left=2.5cm,right=2.5cm,top=2.0cm,bottom=2.5cm}
% \usepackage{CJK}
\usepackage[english]{babel}
\usepackage{amsmath,amsthm}
\usepackage{amsfonts}
\usepackage{algorithm}
\usepackage{algorithmicx}
\usepackage{algpseudocode}
\usepackage{fancyhdr}
% \usepackage{ctex}
\usepackage{array}
\usepackage{listings}
\usepackage{color}
\usepackage{graphicx}
\usepackage{float}
\usepackage{minted}
\usepackage[defaultmono]{droidsansmono}
\usepackage[colorlinks=true,xetex]{hyperref}

\title{
  {\heiti《算法分析与设计》第 $5$ 次作业
    \footnote{要求:1、分析题请用书面化语言给出详细分析过程。}
    }
}
\date{}
\input{personal_info/me.tex}

\definecolor{bg}{rgb}{0.95,0.95,0.95}

\begin{document}
% \begin{CJK}{UTF8}{song}
  \maketitle


  \noindent
  \section*{\bf \color{red}{算法分析题}}

  \noindent
  {\bf 题目1:}用后进先出的 DFS 搜索法找出$n$-皇后问题的一个解。

  \vspace{5pt}
  \noindent
  {\bf 答:}\\ 直接使用 C++ 代码实现。

  \begin{minted}[bgcolor=bg,frame=lines,linenos=true,autogobble,breaklines]{cpp}
      /*  n-皇后解决方法
       *  采用 DFS 算法实现,使用位运算优化
       *  语言标准为 C++11
       *  随文档一并采用 GNU GPL V3 授权
       */

      #include <iostream>  // stdio
      #include <vector>    // std::vector

      using state_t = unsigned long long; // 表示当前状态的数据类型
      // 对于第 i 位,若为 0 则表示第 i+1 列不可放置皇后,否则可放置

      int n; // 题目输入的 n,即皇后个数

      /*  @brief 向屏幕输出计算出的任意一种解,输出格式为 n * n 的
       *  @brief 字符矩阵,从上至下依次表示各行的摆放情况,其中 '.'
       *  @brief 表示空格,'*' 表示放置一个皇后
       *
       *  @param ans 计算出的解序列
       */
      void print_ans(const std::vector<int> &ans)
      {
          for (int x : ans)
          {
              for (int i = 1; i <= n; ++i)
              {
                  std::cout << (i == x ? '*' : '.');
              }
              std::cout << std::endl;
          }
      }

      /*  @brief DFS 求解八皇后问题
       *  @param level 搜索的深度,亦为正在放置的皇后次序
       *  @param left 二进制位表示的已放置的皇后左下方不能放置的格子
       *  @param straight 二进制位表示的已放置的皇后正下方不能放置的格子
       *  @param right 二进制位表示的已放置的皇后右下方不能放置的格子
       *  @return 是否成功找到解。用于在找到任意解后结束算法
       */
      bool dfs(int level, state_t left, state_t straight, state_t right)
      {
          static std::vector<int> ans; // 解序列
          // 已放置成功 n 个皇后,即成功找到一组可行解
          if (level == n + 1)
          {
              print_ans(ans);
              return true;
          }

          int row = left | straight | right; // 本行中不可用的格子
          // 枚举本行的皇后的位置,i 表示第 i+1 行,解序列中记录 i+1 的值
          for (int i = 0; i < n; ++i)
          {
              state_t cur = 1ULL << i;
              // 此处可放置皇后
              if (!(row & cur))
              {
                  ans.push_back(i + 1);
                  // 本行的 left 左移一位即可产生下一行的 left
                  // 注意本行的皇后也会产生一个不可用格子,对后续产生影响
                  // straight, right 同理
                  if (dfs(level + 1, (left | cur) << 1, straight | cur, (right | cur) >> 1))
                  {
                      // 成功找到一个解
                      return true;
                  }
                  ans.pop_back();
              }
          }
          // 未能找到解
          return false;
      }

      int main()
      {
          std::cin >> n;
          dfs(1);
      }
  \end{minted}

  \newpage

  \vspace{10pt}
  \noindent
  {\bf 题目2:}请设计一个回溯算法来搜索一个有向图$G(V,E)$的所有哈密尔顿回路。假定图$G$是用邻接矩阵表示的。

  \vspace{5pt}
  \noindent
  {\bf 答:}\\


  \vspace{10pt}
  \noindent
  {\bf 题目3:}用最大价值先出的分枝限界法找一个图的最大团。我们假定图$G$是用邻接矩阵表示的。

  \vspace{5pt}
  \noindent
  {\bf 答:}\\



% \end{CJK}
\end{document} 