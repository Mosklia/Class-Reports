% \documentclass[12pt,a4paper]{article}
\documentclass[12pt,a4paper]{ctexart}
\usepackage{geometry}
\geometry{left=2.5cm,right=2.5cm,top=2.0cm,bottom=2.5cm}
% \usepackage{CJK}
\usepackage[english]{babel}
\usepackage{amsmath,amsthm}
\usepackage{amsfonts}
\usepackage[ruled,linesnumbered]{algorithm2e}
\usepackage{fancyhdr}
% \usepackage{ctex}
\usepackage{array}
\usepackage{listings}
\usepackage{color}
\usepackage{graphicx}

\theoremstyle{definition}
\newtheorem*{solution}{解}
\newtheorem*{answer}{答}

% Setting for algorithm display
\SetStartEndCondition{ }{}{}%
\SetKwProg{Fn}{}{}{}
\SetKwFunction{Range}{range}%%
\SetKw{KwTo}{in}
\SetKwFor{For}{for}{\string:}{}%
\SetKwIF{If}{ElseIf}{Else}{if}{}{elseif}{else}{}%
\SetKwFor{While}{while}{}{fintq}%
\SetKw{Return}{return}
% \renewcommand{\forcond}{$i$ \KwTo\Range{$n$}}
% Setting for algorithm display end

\begin{document}
% \begin{CJK}{UTF8}{song}

\title{
{\heiti《算法分析与设计》第 $2$ 次作业
\footnote{要求:1、分析题请用书面化语言给出详细分析过程。}
}
}
\date{}

\author{
    姓名:\underline{陈宇轩}~~~~~~
    学号:\underline{71120226}~~~~~~
    成绩:\underline{~~~~~~~~~~~~~~~~~~}
}

\maketitle

\noindent
\section*{\bf \color{red}{算法分析题}}

\noindent
{\bf 题目1:}求以下递推关系。
\begin{enumerate}
    \item[(1)]  $T(n)=4T(n/2)+n^{2}\log_{2}{n}$
    \item[(2)]  $T(n)=2T(\sqrt{n})+\log_{2}{n}$
\end{enumerate}

\vspace{5pt}
\noindent
% {\bf 答:}
% \begin{proof}
    
% \end{proof}
\begin{solution}
    (1) 令 $n = 2^k$,则可得:
    \begin{align*}
        T(2^k) &= 4^1 T(2^{k-1}) + k2^{2k} \\
        &= 4^2 T(2^{k-2}) + k4^k + (k-1)4^k \\
        &= \cdots \\
        &= 4^k T(1) + \sum_{i=1}^{k}i4^k \\
        &= 2^{2k} + \frac{k(k+1)2^{2k}}{2} \\
        &= \Theta((k2^k)^2) 
    \end{align*}
    将 $k = \log_2 n$ 代回即得:
    \[
        T(n) = \Theta(n^2\log^2_2n)
    \]

    (2) 令 $n = 2^{2^k}$,则可得:
    \begin{align*}
        T(2^{2^k}) &= 2T(2^{2^{k-1}}) + 2^k \\
        &= 2^2T(2^{2^{k-2}}) + 2^k + 2^k \\
        &= \cdots \\
        &= 2^kT(2) + k2^k \\
        &= \Theta(k2^k)
    \end{align*}
    将 $k = \log_2 \log_2 k$ 代回即得:
    \[
        T(n) = \Theta(\log_2k \cdot \log_2\log_2k)
    \]
\end{solution}

\begin{answer}
    \begin{enumerate}
        \item[(1)] $T(n) = \Theta(n^2\log^2_2n)$;
        \item[(2)] $T(n) = \Theta(\log_2k \cdot \log_2\log_2k)$.
    \end{enumerate}
\end{answer}

\vspace{10pt}
\noindent
{\bf 题目2:}用分治法设计一个算法同时找出数组$A[1..n]$中的最大和第二大的数,$n\geq2$,并分析所需的比较次数(请写出关键思路或伪代码)。

\vspace{5pt}
\noindent
\begin{solution}
    如算法 \ref{algo:t2} 所示。
    \begin{algorithm}[hb]
        \label{algo:t2}
        \SetAlgoNoLine
        \AlgoDontDisplayBlockMarkers
        \SetAlgoNoEnd
        \caption{分治法寻找前二大的两个数}
        \SetKwFunction{BProcess}{BINARY-PROCESS}
        \Fn{\BProcess{A}}{
            \KwIn{The array $A$[1..n]}
            \KwOut{A pair of 2 numbers, ($fst$, $sed$), that stands for the first- and second-greatest value in the array $A$}
            \If{$A.length$ = 1} {
                \Return ($A[1]$, $-\infty$) \\
            }
            \ElseIf{$A.length$ = 2} {
                \Return ($\max\{A[1], A[2]\}$, $\min\{A[1], A[2]\}$) \\
            }
            $mid$ = $\lfloor(1 + A.length) / 2\rfloor$ \\
            $result_\mathrm{left}$ = \BProcess{$A[1..mid]$} \\
            $result_\mathrm{right}$ = \BProcess{$A[mid+1..A.length]$} \\
            \tcc{Maximum value only exists in \{$result_\mathrm{left}$[1], $result_\mathrm{right}$[1]\}}
            $fst$ = $\max\{result_\mathrm{left}[1], result_\mathrm{right}[1]\}$ \\
            $sed$ = $\max\{\min\{result_\mathrm{left}[1], result_\mathrm{right}[1]\}, \max\{result_\mathrm{left}[2], result_\mathrm{right}[2]\}\}$ \\
            \Return ($fst$, $sed$)
        }
    \end{algorithm}
\end{solution}


\vspace{10pt}
\noindent
{\bf 题目3:}设$A$和$B$是两个$n \times n$矩阵。众所周知,计算其乘积$C=AB$通常需要做$\Theta(n^{3})$乘法和加法。基于分治法的$Strassen$算法可以改进这个复杂度。下面是这个算法。为简化起见,我们假设$n=2^{k}$。 \\
{\bf Strassen's algorithm (A, B, n)}
\begin{enumerate}
    \item[(a)]  将$A$和$B$各自划分为4个$n/2 \times n/2$ 的矩阵,如下。\\
        $$
            A=
            \begin{bmatrix}
                A_{11} & A_{12} \\
                A_{21} & A_{22}
            \end{bmatrix}
            ,
            \qquad \qquad
            B=
            \begin{bmatrix}
                B_{11} & B_{12} \\
                B_{21} & B_{22}
            \end{bmatrix}
        $$
    \item[(b)]  按下面公式递归计算出7个$n/2 \times n/2$ 的矩阵。
        \begin{align*}
             & P=(A_{11}+A_{22})(B_{11}+B_{22}); \\
             & Q=(A_{21}+A_{22})B_{11};          \\
             & R=A_{11}(B_{12}-B_{22});          \\
             & S=A_{22}(B_{21}-B_{11});          \\
             & T=(A_{11}+A_{22})B_{22};          \\
             & U=(A_{21}-A_{11})(B_{11}+B_{12}); \\
             & V=(A_{12}-A_{22})(B_{21}+B_{22});
        \end{align*}
    \item[(c)]  按下面公式计算出4个$n/2 \times n/2$ 的矩阵。
        \begin{align*}
             & C_{11}=P+S-T+V; \\
             & C_{12}=R+T;     \\
             & C_{21}=Q+S;     \\
             & C_{22}=P+R-Q+U;
        \end{align*}
    \item[(d)]  输出结果如下。\\
        $$
            C=AB=
            \begin{bmatrix}
                C_{11} & C_{12} \\
                C_{21} & C_{22}
            \end{bmatrix}
        $$
\end{enumerate}
请分析$Strassen$算法的复杂度。不必证明其正确性。

\vspace{5pt}
\noindent
\begin{solution}
    对于规模为 $n = 2^k$ 的方阵而言,Strassen 算法需要调用 $7$ 次规模为 $\frac{n}{2} = 2^{k-1}$ 的矩阵乘法,以及 $18$ 次规模相同的矩阵加法,
    故可知:

    \begin{align*}
        T(n) &= 7T(\frac{n}{2})+18(n-1)^2 \\
        &= 7T(\frac{n}{2})+18n^2-36n+18 \\
        &= 7T(\frac{n}{2})+O(n^2)
    \end{align*}

    令$a = 7, b = 2$,取 $\varepsilon = \log_2 7 - 2 > 0$,则 $f(n) = O(n^{\log_b a - \varepsilon})$,由主定理知,
    \[
        T(n) = \Theta(n^{\log_b a}) = \Theta(n^{\log_2 7})
    \]
\end{solution}
\begin{answer}
        $T(n) = \Theta(n^{\log_2 7})$
\end{answer}

% \end{CJK}
\end{document}