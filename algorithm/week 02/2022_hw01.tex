\documentclass[12pt,a4paper]{ctexart}
\usepackage{geometry}
\geometry{left=2.5cm,right=2.5cm,top=2.0cm,bottom=2.5cm}
% \usepackage{CJK}
\usepackage[english]{babel}
\usepackage{amsmath,amsthm}
\usepackage{amsfonts}
\usepackage[longend,ruled,linesnumbered]{algorithm2e}
\usepackage{fancyhdr}
% \usepackage{ctex}
\usepackage{array}
\usepackage{listings}
\usepackage{color}
\usepackage[colorlinks=true,bookmarks=true,bookmarksnumbered=true]{hyperref}

\renewcommand{\theenumi}{(\arabic{enumi})}

\begin{document}
% \begin{CJK}{UTF8}{song}

\title{
  {
    \heiti《算法分析与设计》第 {1} 次作业
    \footnote{要求:1、分析题请用书面化语言给出详细分析过程。}
  }
}
\date{}

\author{
姓名:\underline{陈宇轩}~~~~~~
学号:\underline{71120226}~~~~~~
成绩:\underline{~~~~~~~~~~~~~~~~~~}
}

\maketitle

\noindent
\section*{\bf \color{red}{算法分析题}}

\noindent
{\bf 题目1:}求下列函数的$\Theta$渐进表达式
\begin{enumerate}
  \item[(1)]  $n^2/10+2^n$
  \item[(2)]  $21+1/n$
  \item[(3)]  $10\log{}{3^n}$
\end{enumerate}

\vspace{5pt}
\noindent
{
  \textbf{答:}

  (1) 取 $C_1 = 1, C_2 = 1000$,则当 $n \geq 1$ 时,显然有 $C_1 \cdot 2^n \leq n^2/10+2^n \leq C_2 \cdot 2^n$,即 $C_1 \cdot 2^n \leq T(n) \leq C_2 \cdot 2^n$;
      故函数 $n^2/10+2^n$ 的渐进表达式 $n^2/10+2^n = \Theta(2^n)$。

  (2) 取 $C_1 = 21, C_2 = 22, f(n) = 1$,则当 $n \geq 1$ 时,$C_1f(n) = 21 \leq 21 + \frac{1}{n} \leq 22 = C_2f(n)$,即 $C_1f(n) \leq T(n) \leq C_2f(n)$;
      故函数 $21 + \frac{1}{n}$ 的渐进表达式为 $21 + \frac{1}{n} = \Theta(1)$。

  (3) $T(n) = 10 \log{3^n} = (10 \log3) n$,故令 $C_1 = \log3, C_2 = 100\log3, f(n) = n$,则当 $n \geq 1$ 时有 $C_1 \leq 10\log3 \leq C_2$,可得 $C_1n \leq (10\log3)n \leq C_2n$,即$C_1f(n) \leq T(n) \leq C_2f(n)$,
      故函数 $10\log{}{3^n}$ 的渐进表达式为 $10\log{}{3^n} = \Theta(n)$。
}


\vspace{10pt}
\noindent
{\bf 题目2:}对以下每个级数和$T(n)$,用$\Theta$记号表示与其同阶的只含一项的函数。
\begin{enumerate}
  \item[(1)]  $T(n)=\sum\limits_{k=1}^n\frac{k^3\lg k+1}{k^2+k+1}$
  \item[(2)]  $T(n)=\sum\limits_{k=1}^n\frac{\sqrt{k}+k\lg k+8}{3k^2\lg k+5k+1}$
\end{enumerate}

\vspace{5pt}
\noindent
{
  \textbf{答:}

  (1) $T(n)=\sum\limits_{k=1}^n\frac{k^3\lg k+1}{k^2+k+1} = \sum\limits_{k=1}^n(\frac{k^3\lg k}{k^2+k+1}+\frac{1}{k^2+k+1}) = \sum\limits_{k=1}^n\frac{k^3\lg k}{k^2+k+1}+\sum\limits_{k=1}^n\frac{1}{k^2+k+1}$
}


\vspace{10pt}
\noindent
{\bf 题目3:}用$\Theta$记号表示程序\ref{algo:t3}中语句$x \gets x+1$被执行的次数的估计.
\begin{algorithm}[h]
\caption{一个算法实例}
\label{algo:t3}
\For{i=1 \emph{\KwTo} n}{
    \For{j=i \emph{\KwTo} 3i}{
    $x \gets x+1$\;
    }
	}
\end{algorithm}

\vspace{5pt}
\noindent
{
  \textbf{答:}
  易知语句 $x \gets x+1$ 的执行次数为:
  \begin{align*}
      T(n) &= \sum_{i=1}^{n} \sum_{j=i}^{3i} 1 \\
           &= \sum_{i=1}^{n} 2i \\
           &= n^2 + n
  \end{align*}

  令 $C_1 = 1, C_2 = 2, f(n) = n^2$,则当 $n \geq 1$ 时,有 $0 \leq n \leq n^2$。在不等式各项加上一个 $n^2$,得

  \[
    1 \cdot n^2 \leq n^2 + n \leq 2 \cdot n^2
  \]

  即

  \[
    C_1 \cdot f(n) \leq T(n) \leq C_2 \cdot f(n)
  \]

  故语句 $x \gets x+1$ 的执行次数为 $T(n) = \Theta(n^2)$。
}

% \end{CJK}
\end{document} 