\documentclass[12pt,a4paper]{ctexart}
\usepackage{geometry}
\geometry{left=2.5cm,right=2.5cm,top=2.0cm,bottom=2.5cm}
% \usepackage{CJK}
\usepackage[english]{babel}
\usepackage{amsmath,amsthm}
\usepackage{amsfonts}
% \usepackage[longend,ruled,linesnumbered]{algorithm2e}
\usepackage[indLines=false,beginComment=//~,beginLComment=//~,endLComment=]{algpseudocodex}
\usepackage{algorithm}
\usepackage{fancyhdr}
% \usepackage{ctex}
\usepackage{array}
\usepackage{listings}
\usepackage{color}
\usepackage{graphicx}
\usepackage{subfig}
\usepackage{diagbox}
\usepackage[xetex,colorlinks=true]{hyperref}

% \tikzset{algpxIndentLine/.style={draw=white}}

\algnewcommand{\To}{\textbf{to} }
\algnewcommand{\DownTo}{\textbf{downto} }

\begin{document}
% \begin{CJK}{UTF8}{song}

\title{
  {\heiti《算法分析与设计》第 {$3$} 次作业
    \footnote{要求:1、分析题请用书面化语言给出详细分析过程。}
    }
}
\date{}

\author{
姓名:\underline{陈宇轩}~~~~~~
学号:\underline{71120226}~~~~~~
成绩:\underline{~~~~~~~~~~~~~~~~~~}
}

\maketitle

\noindent
\section*{\bf \color{red}{算法分析题}}

\noindent
{\bf 题目1:}一个字符串中的连续的一段字符序列称为该字符串的一个子串。例如$school$中$cho$就是一个子串。子串和子序列的区别是,子序列在原序列中的位置不一定连续,而子串必须连续。两个字符串都含有的子串称为它们的公共子串。例如,$university$和$anniversary$就含有许多公共子串,例如$ni,niv,y,er$等,但最长的 一个是$nivers$。假设有字符串$X=x_1x_2...x_n$和$Y=y_1y_2...y_m$,请设计一个基于动态规划的$O(mn)$算法来找出$X$和$Y$的最长公共子串。(要写出最长公共子串在字符串$X$的起止位置 )

\textbf{解:} 设状态 $f[i, j]$ 表示以 $X[i], Y[j]$ 为结尾的公共子串的最大长度,则状态转移方程如下:

\begin{align*}
  f[i, j] = \begin{cases}
    f[i - 1, j - 1] + 1, & X[i] = Y[j] \\
    0, & \text{otherwise}
  \end{cases}
\end{align*}

迭代计算以上各状态的值,同时更新答案即可。

\vspace{5pt}
\noindent
{\bf 答:}见算法 \ref{algo:t1}。

\begin{algorithm}
  \caption{寻找最长公共子串}\label{algo:t1}
  \begin{algorithmic}[1]
    \Require 两个字符串 $X, Y$,字符集有限
    \Ensure 一个有序整数三元组$(posX, posY, len)$,表示 $X, Y$ 的最长公共子串在 $X$ 与 $Y$ 的起始位置及长度
    \Function{FindLongestCommonSubstr}{$X, Y$}
      \LComment {初始化答案计数器}
      \State $posX \gets 0$
      \State $posY \gets 0$
      \State $len \gets 0$
      \LComment {初始化边界条件}
      \For {$i \gets 1$ \To $X.length$}
        \State $f[i, 1] \gets X[i] = Y[1]$ ? $1$ : $0$
        \State \Call {UpdateAnswer}{$i, 1$}
      \EndFor
      \For {$i \gets 2$ \To $Y.length$}
        \State $f[1, i] \gets X[1] = Y[i]$ ? $1$ : $0$
        \State \Call {UpdateAnswer}{$1, i$}
      \EndFor
      \LComment {迭代计算}
      \For {$i \gets 2$ \To $X.length$}
        \For {$j \gets 2$ \To $Y.length$}
          \If {$X[i] = Y[j]$}
            \State $f[i, j] \gets f[i - 1, j - 1] + 1$
          \Else
            \State $f[i, j] \gets 0$
          \EndIf
          \State \Call {UpdateAnswer}{$i, j$}
        \EndFor
      \EndFor
      \State \Return ($posX, posY, len$)
      \Statex
      \LComment {辅助过程,更新答案}
      \Procedure{UpdateAnswer}{$i,j$}
        \If {$f[i, j] > len$}
          \State $posX \gets i$
          \State $posY \gets j$
          \State $len \gets f[i, j]$
        \EndIf
      \EndProcedure
    \EndFunction
  \end{algorithmic}
\end{algorithm}

\clearpage

\vspace{10pt}
\noindent
{\bf 题目2:}顺序放好的$n$根钢管的重量各为$W[i], 1 \leq i \leq n$。
我们需要把它们依照顺序焊成一根钢管,但每次焊接可任意选两根相邻的钢管来焊接。每次焊接的代价与被焊两段钢管中的较重的一根的重量成正比。为简单起见,就把代价定为被焊两段钢管中的较重的一根的重量。例如,$W[1]=5, W[2]=1, W[3]=2, $ 如果先把$W[1]$和$W[2]$焊好,代价为$5$,再把$W[3]$上,又要代价$6$,总代价是$11$。但如果先焊$W[2]$和$W[3]$,再焊$W[1]$,则总代价为$2+5=7$。
\begin{enumerate}
  \item[(a)]  用动态规划的方法设计一个算法计算出最优的焊接顺序,使总代价最小。
  \item[(b)]  应用你上面的算法求出一下$5$根钢管的最优焊接顺序和总代价:$W[1]=6, W[2]=2, W[3]=7. W[4]=5, W[5]=8$。
\end{enumerate}

\textbf{解:} 典型的区间 DP。设状态 $f[i, j]$ 表示“将区间 $[i, j]$ 的全部钢管焊接到一起的最小代价”,则状态转移方程为:

\begin{equation}
  f[i, j] = \min_{k=i,i+1,i+2...j-1}\{f[i, k] + f[k + 1, j] + \max\{\sum_{o=1}^kW[o], \sum_{o=k+1}^jW[o]\}\} \label{formula:t2}
\end{equation}

迭代计算即可。注意计算中应当按照区间长度(即 $j-i$ 的大小)作为第一关键字升序迭代,并且可以利用前缀和来降低求 $\sum W[o]$ 带来的时间复杂度升高。

\vspace{5pt}
\noindent
{\bf 答:} (a) 见算法 \ref{algo:t2};

\begin{algorithm}
  \caption{合并钢管} \label{algo:t2}
  \begin{algorithmic}[1]
    \Require 一个非负数列 $W[1..n]$,表示各钢管的重量。
    \Ensure 一个非负整数 $res$,表示将所有钢管焊接的总代价。
    \Function {WeldPipes}{$W[1..n]$}
      \LComment {求前缀和数组 $sum[1..n]$,即 $sum[i] = \sum_{k=1}^iW[k]$}
      \State $sum[1] \gets W[1]$
      \For {$i \gets 2$ \To $n$}
        \State $sum[i] \gets sum[i - 1] + W[i]$
      \EndFor
      \LComment {初始化动态规划数组 $f[1..n][1..n]$}
      \For {$i \gets 1$ \To $n$}
        \For {$j \gets 1$ \To $n$}
          \State $f[i][j] \gets i = j$ ? 0 : $+\infty$
        \EndFor
      \EndFor
      \LComment {迭代计算}
      \For {$k \gets 1$ \To $n - 1$} \Comment {区间长度 $-1$,对应式 \eqref{formula:t2} 的 $j-i$}
        \For {$i \gets 1$ \To $n - k$} \Comment {区间起点,对应式 \eqref{formula:t2} 的 $i$}
          \For {$j \gets i$ \To $i + k - 1$} \Comment {区间断点,对应式 \eqref{formula:t2} 的 $k$}
            \LComment{$cost$: 本次划分中更重的那个子段的重量}
            \State $cost \gets \max\{$ \Call {SegWeight}{$i,j$}, \Call {SegWeight}{$j+1,i+k$} $\}$ 
            \State $f[i][i + k] \gets \min\{f[i][i + k], f[i][j] + f[j + 1][i + k] + cost \}$
          \EndFor
        \EndFor
      \EndFor
      \State \Return $f[1][n]$
      \Statex
      \LComment {辅助函数:根据前缀和求区间$[s,t]$的重量和}
      \Function {SegWeight}{$s,t$}
        \If {$s = 1$}
          \State \Return $sum[t]$
        \Else
          \State \Return $sum[t] - sum[s - 1]$
        \EndIf
      \EndFunction
    \EndFunction
  \end{algorithmic}
\end{algorithm}

(b) 计算过程用到的数据见表 \ref{table:t2}。

\begin{table}[hp]
  \centering
  \caption{合并钢管计算表}
  \label{table:t2}
  \subfloat[前缀和数组 $sum$]{
    \begin{tabular}[h]{|*{6}{c|}}
      \hline
      $i$ & 1 & 2 & 3 & 4 & 5 \\ \hline
      $W[i]$ & 6 & 2 & 7 & 5 & 8 \\ \hline
      $sum[i]$ & 6 & 8 & 15 & 20 & 28 \\ \hline
    \end{tabular}
  }
  \quad
  \subfloat[状态矩阵 $f$ 计算表]{
    \begin{tabular}{|*{6}{c|}}
      \hline
      \diagbox[dir=NW]{$i$}{$f[i,j]$}{$j$} & 1 & 2 & 3 & 4 & 5 \\ \hline
      1 & 0 & 6 & 14 & 25 & 37 \\ \hline
      2 & - & 0 & 7 & 16 & 28 \\ \hline
      3 & - & - & 0 & 7 & 19 \\ \hline
      4 & - & - & - & 0 & 8 \\ \hline
      5 & - & - & - & - & 0 \\ \hline
    \end{tabular}
  }
\end{table}

由表 \ref{table:t2} 的数据可知,最小总代价为 $37$。

\clearpage

\vspace{10pt}
\noindent
{\bf 题目3:}假设我们有$n$个活动要申请使用大礼堂:$S=\{a_1,a_2,...,a_n\}$。礼堂从时间$t=0$起可以安排。活动$a_i(1 \leq i \leq n)$需要连续使用的时间是$t_i$并且在时刻$d_i$前结束,这里有$0<t_i \leq d_i< \infty$。也就是说它必须被安排在时刻$d_i-t_i$前开始,否则不可能按时完成。因为在任何时刻,两个活动不能同时使用礼堂,所以我们只能满足一部分活动,但我们希望能满足尽量多的活动。
\begin{enumerate}
  \item[(a)]  请用动态规划的方法设计一个$O(n^2)$算法找出几何$S$中最大的一个活动子集并作出它们可以不冲突地使用大礼堂的调度。(提示:先把几何$S$中活动按它们的结束时间$d_i(1 \leq i \leq n)$排序。可假定$d_1 \leq d_2 \leq ... \leq d_n$。然后,定义子集$S_i=\{a_1,a_2,...,a_i\}, 1 \leq i \leq n$。按照动态规划的原理,先考虑$S_1$,再考虑$S_2,...$,最后考虑$S_n$。我们定义$M(i,j)$为安排集合$S_i$中$j$个活动所需的最短时间。如果集合$S_i$中根本没有$j$个活动,或者不可能找出这$j$个活动,则置$M(i,j)= \infty$。假设我们已算出$M(i,j),1 \leq i \leq k, 1 \leq j \leq n$,我们该怎样算$M(i+1,j),1 \leq j \leq n$?)
  \item[(b)]  请用上面的算法找出下面7个活动的最优解。
    \begin{table}[htb]   
    \begin{center}   
    \caption{活动参数}  
    \label{table:1} 
    \begin{tabular}{|c|c|c|c|c|c|c|c|}   
    \hline   \textbf{A[i]} & 1 & 2 & 3 & 4 & 5 & 6 & 7 \\   
    \hline   \textbf{T[i]} & 4 & 4 & 2.5 & 4.5 & 6 & 3 & 5  \\ 
    \hline   \textbf{D[i]} & 5 & 8 & 7.5 & 11.5 & 14 & 12 & 15  \\  
    \hline   
    \end{tabular}   
    \end{center}   
    \end{table}
\end{enumerate}

\textbf{解:} 显然,应该让每个活动占用礼堂的时间尽可能短,即对于任务 $a_i$,我们应令其占用礼堂的时间恰好为 $t_i$。并且,每个活动之间不应该留有时间间隔。

注意到题目并没有规定各个活动间的相对顺序。对于每个活动,只要能在规定时间前结束即可。于是设状态 $f[i, j]$ 表示考虑前 $i$ 个活动,并且安排了 $j$ 个活动的总最短时间。
则状态转移方程为:

\begin{equation}
  f[i, j] = \begin{cases} \label{formula:t3}
    \min\{f[i-1, j-1] + t[i], f[i-1, j]\}, & f[i-1,j-1] + t[i] < d[i] \\
    f[i-1,j], & f[i-1,j-1] + t[i] \geq d[i]
  \end{cases}
\end{equation}

然后考虑对活动进行排序。可以基于交换法证明,应该将活动将结束时间 $d_i$ 作为第一关键字升序,持续时间 $t_i$ 作为第二关键字升序排序。排序以后按照式 \eqref{formula:t3} 迭代计算即可。

最终答案就是最大的正整数 $i \in [1, n], \text{使得} f[n, i] \neq +\infty$。

\vspace{5pt}
\noindent
{\bf 答:} (1) 见算法 \ref{algo:t3};

\begin{algorithm}
  \caption{活动调度算法} \label{algo:t3}
  \begin{algorithmic}[1]
    \Require 一个结构体序列 $act[1..n]$,表示各个活动的信息,其元素包含三个字段 $a, t, d$(其中字段 $a$ 仅在记录方案过程中使用),含义见题面;
    \Ensure 一个正整数 $res$,表示最多可以安排的活动数量,和一个正整数列 $sol[1..res]$,表示最终安排的活动及其顺序。
    \Function {ActivitySchedule}{$act[1..n]$}
      \State 对 $act[1..n]$ 中的元素按照字段 $t$ 为第一关键字,字段 $d$ 为第二关键字升序排序
      \State 初始化状态矩阵 $f[0..n][0..n]$,使得 $f[0, x] = +\infty, f[y, 0] = 0 (x = 1,2,\ldots,n, y=0,1,\ldots,n)$
      \LComment {迭代计算}
      \For {$i \gets 1$ \To $n$}
        \For {$j \gets 1$ \To $i$}
          \State $newCost = f[i-1, j-1] + act[i].d$
          \If {$newCost < act[i].t$ \textbf{and} $newCost < f[i-1,j]$}
            \State $f[i, j] \gets newCost$
            \State $status[i, j] \gets 0$ \Comment {$status[i, j]$ 表示当前状态从 $f[i-1,j-1]$($0$) 还是从 $f[i-1,j]$($1$) 转移而来}
          \Else
            \State $f[i, j] \gets f[i-1, j]$
            \State $status[i, j] \gets 1$
          \EndIf
        \EndFor
      \EndFor
      \State $res \gets$ 最大的 $k$,使得 $f[n,k] \neq +\infty$
      \LComment {通过 $status$ 来倒推出实际方案}
      \State $(x, y) \gets (n, res)$
      \While {$x > 0$}
        \If {$status[x, y] = 0$}
          \State $sol \gets [x] + sol$ \Comment{将 $x$ 插入到方案序列 $sol$ 的开头}
          \State $(x, y) \gets (x-1, y-1)$
        \Else
          \State $(x, y) \gets (x-1, y)$
        \EndIf
      \EndWhile
      \State \Return ($res$, $sol$)
    \EndFunction
  \end{algorithmic}
\end{algorithm}

(2) 计算中用到的矩阵如表 \ref{table:t3} 所示。

\begin{table}
  \centering
  \caption{活动调度计算表}
  \label{table:t3}
  \subfloat[$act$ 序列排序结果]{
    \begin{tabular}[h]{|*{8}{c|}}
      \hline
      $i$ & 1 & 2   & 3 & 4    & 5  & 6  & 7 \\ \hline
      $a$ & 1 & 3   & 2 & 4    & 6  & 5  & 7 \\ \hline
      $t$ & 4 & 2.5 & 4 & 4.5  & 3  & 6  & 5 \\ \hline
      $d$ & 5 & 7.5 & 8 & 11.5 & 12 & 14 & 15 \\ \hline
    \end{tabular}
  }
  \\
  \subfloat[矩阵 $f$ 计算表]{
    \begin{tabular}[h]{|*{8}{c|}}
      \hline
      \diagbox[dir=NW]{$i$}{$f[i,j]$}{$j$} & 1 & 2 & 3 & 4 & 5 & 6 & 7 \\ \hline
      1 & 4 & $+\infty$ & $+\infty$ & $+\infty$ & $+\infty$ & $+\infty$ & $+\infty$ \\ \hline
      2 & 2.5 & 6.5 & $+\infty$ & $+\infty$ & $+\infty$ & $+\infty$ & $+\infty$ \\ \hline
      3 & 2.5 & 6.5 & $+\infty$ & $+\infty$ & $+\infty$ & $+\infty$ & $+\infty$ \\ \hline
      4 & 2.5 & 6.5 & 11 & $+\infty$ & $+\infty$ & $+\infty$ & $+\infty$ \\ \hline
      5 & 2.5 & 5.5 & 9.5 & $+\infty$ & $+\infty$ & $+\infty$ & $+\infty$ \\ \hline
      6 & 2.5 & 5.5 & 9.5 & $+\infty$ & $+\infty$ & $+\infty$ & $+\infty$ \\ \hline
      7 & 2.5 & 5.5 & 9.5 & 14.5 & $+\infty$ & $+\infty$ & $+\infty$ \\ \hline
    \end{tabular}
  }
  \quad
  \subfloat[矩阵 $status$ 计算表]{
    \begin{tabular}[h]{|*{8}{c|}}
      \hline
      \diagbox[dir=NW]{$i$}{$status[i,j]$}{$j$} & 1 & 2 & 3 & 4 & 5 & 6 & 7 \\ \hline
      1 & 0 & - & - & - & - & - & - \\ \hline
      2 & 0 & 0 & - & - & - & - & - \\ \hline
      3 & 1 & 1 & 1 & - & - & - & - \\ \hline
      4 & 1 & 1 & 0 & 1 & - & - & - \\ \hline
      5 & 1 & 0 & 0 & 1 & 1 & - & - \\ \hline
      6 & 1 & 1 & 1 & 1 & 1 & 1 & - \\ \hline
      7 & 1 & 1 & 1 & 0 & 1 & 1 & 1 \\ \hline
    \end{tabular}
  }
\end{table}

由表 \ref{table:t3} 可得,$res = 4$,$sol = \{1, 3, 6, 7\}$。

% \end{CJK}
\end{document} 