\documentclass[12pt,a4paper]{ctexart}
\usepackage{geometry}
\geometry{left=2.5cm,right=2.5cm,top=2.0cm,bottom=2.5cm}
% \usepackage[english]{babel}
\usepackage{amsmath,amsthm}
\usepackage{amsfonts}
\usepackage[longend,ruled,linesnumbered]{algorithm2e}
\usepackage{fancyhdr}
\usepackage{array}
\usepackage{listings}
\usepackage{color}
\usepackage{graphicx}
\usepackage{minted}
\usepackage{float}
\usepackage[defaultmono]{droidsansmono}

\graphicspath{{pics/}}
\ctexset{today=small}
\definecolor{codebg}{rgb}{0.95,0.95,0.95}

\input{personal_info/info.tex}

\begin{document}
    \begin{titlepage}
        \heiti
        \vspace*{64pt}
        \begin{center}
            \fontsize{48pt}{0} 算法设计与分析\\
            \vspace*{36pt}
            \fontsize{48pt}{0}{实\quad 验\quad 报\quad 告}\\
            \vspace*{48pt}
            \LARGE(2021\~{}2022 学年度\qquad 第 3 学期)\\
            \vspace*{48pt}
        
            \LARGE 实验名称\ \ \underline{\makebox[200pt]{\ExamTitle}}\\
            \LARGE 实验地点\ \ \underline{\makebox[200pt]{\ExamAddr}}\\
            \LARGE 实验日期\ \ \underline{\makebox[200pt]{\today}}\\
            \LARGE 学生姓名\ \ \underline{\makebox[200pt]{\MyName}}\\
            \LARGE 学生学号\ \ \underline{\makebox[200pt]{\MySID}}\\
            \LARGE 指导教师\ \ \underline{\makebox[200pt]{\TeacherName}}\\
            \vspace*{48pt}
            
            \LARGE 东南大学\quad  计软智学院 \quad 制
        \end{center}
    \end{titlepage}

\title{
  {\heiti \textbf{实验一\ 分治法}
    \footnote{要求:1、分析题请用书面化语言给出详细分析过程。2、实验请统一使用ex0*-学号-姓名的命名格式,latex版本请附上源代码并打包提交。}
    }
}
\date{}

\maketitle

\section*{\bf \color{black}{一、实验目的及意义}}
\noindent
\begin{enumerate}
	\item[(1)]  掌握分治法的基本思想、求解问题的基本步骤;
	\item[(2)]  掌握分治算法的时间复杂度;
	\item[(3)]  掌握快速幂算法、快速排序等高效算法。
\end{enumerate}

\vspace{5pt}

\section*{二、实验内容与结果}
\subsection*{题目1:快速排序算法}
\paragraph{题目内容}
\subparagraph{题目描述}
\begin{itemize}
    \item 输入任意规模的一维数组a[0,1,...,n-1],采用分治法实现一个快速排序算法,实现其从小到大的排序。
    \item \textbf{注意}:直接调用排序库函数实现的不得分。
\end{itemize}

\subparagraph{输入格式}
    \begin{itemize}
        \item 输入一行,需要排序的数组,数组元素为正整数,数组大小为$n(1 \leq n \leq 100)$。
    \end{itemize}
\subparagraph{输出格式}
    \begin{itemize}
        \item 输出一行,已排好序的数组。
    \end{itemize}

\subparagraph{输入输出样例}
如下
\begin{table}[htb]   
    \begin{center}   
    \begin{tabular}{|c|c|}   
    \hline   \textbf{Input 1} &11 \ 6 \ 3 \ 7 \ 8 \ 5 \ 1 \ 4 \ 2 \ 4 \ 9 \ 10 \\   
    \hline   \textbf{Output 1} &1 \ 2 \ 3 \ 4 \ 4 \ 5 \ 6 \ 7 \ 8 \ 9 \ 10 \ 11  \\ 
    \hline   
    \end{tabular}   
    \end{center}   
\end{table}

\vspace{5pt}

\paragraph{实验环境}
\begin{itemize}
    \item 程序设计语言:C++
    \item 编程环境:
    \begin{itemize}
        \item 编辑器:Visual Studio Code (1.67.0)
        \item 编译器:g++ (GCC) 11.2.0
        \item 操作系统:ArchLinux 5.17.5-zen1-1-zen (64-bit)
    \end{itemize}
\end{itemize}

\vspace{5pt}

\paragraph{解答} 时间复杂度:$O(n \log n)$

源码:
\inputminted[bgcolor=codebg,frame=lines,autogobble,linenos=true,breaklines]{cpp}{src/t1.cpp}

\vspace{5pt}

\paragraph{实验结果}
(可附上截图)

\newpage

\subsection*{题目2:快速幂算法}
\paragraph{题目内容}
\subparagraph{题目描述}
\begin{itemize}
    \item 采用分治法求$3^{K}$的最后n位数。 其中0<n<=8,1<K<$10^{8}$。
\end{itemize}

\subparagraph{输入格式}
    \begin{itemize}
        \item 输入一行,$k$和$10^n$,中间用空格隔开。
    \end{itemize}

\subparagraph{输出格式}
    \begin{itemize}
        \item 输出一行,一个整数表示答案。
    \end{itemize}
    
\subparagraph{输入输出样例}
如下
\begin{table}[htb]   
    \begin{center}   
    \begin{tabular}{|c|c|}   
    \hline   \textbf{Input 1} &8 \ 10000 \\   
    \hline   \textbf{Output 1} &6561 \\ 
    \hline   \textbf{Input 2} &12345 \ 1000 \\
    \hline   \textbf{Output 2} &643 \\ 
    \hline   
    \end{tabular}   
    \end{center}   
\end{table}


\vspace{5pt}

\paragraph{实验环境}
\begin{itemize}
    \item 程序设计语言:C++
    \item 编程环境:
    \begin{itemize}
        \item 编辑器:Visual Studio Code (1.67.0)
        \item 编译器:g++ (GCC) 11.2.0
        \item 操作系统:ArchLinux 5.17.5-zen1-1-zen (64-bit)
    \end{itemize}
\end{itemize}

\vspace{5pt}

\paragraph{解答} 时间复杂度:$\Theta(\log K)$

源码:
\inputminted[bgcolor=codebg,frame=lines,autogobble,linenos=true,breaklines]{cpp}{src/t2.cpp}

\vspace{5pt}

\paragraph{实验结果}
(可附上截图)

\newpage

\subsection*{题目3:分治法-数数}
\paragraph{题目内容}
\subparagraph{题目描述}

\begin{itemize}
    \item 输入n个元素构成一个数组。要求返回n个数字,每个数字的含义是:原数组中每个元素右边比它小的个数。
    \item \textbf{注意}:需要采用分治法实现。
\end{itemize}

\subparagraph{输入格式}
    \begin{itemize}
        \item 输入数据只有一行,表示原数组,大小为$N(2 \leq N \leq 10)$
    \end{itemize}

\subparagraph{输出格式}
    \begin{itemize}
        \item 针对原数组,输出一行数字,表示原数组中每个元素右边比它小的个数。
    \end{itemize}
    

\subparagraph{输入输出样例}
如下:
\begin{table}[htb]   
    \begin{center}   
    \begin{tabular}{|c|c|}   
    \hline   \textbf{Input 1} &5 \ 2 \ 6 \ 1\\   
    \hline   \textbf{Output 1} &2 \ 1 \ 1 \ 0 \\ 
    \hline   
    \end{tabular}   
    \end{center}   
\end{table}

\vspace{5pt}

\paragraph{实验环境}
\begin{itemize}
    \item 程序设计语言:C++
    \item 编程环境:
    \begin{itemize}
        \item 编辑器:Visual Studio Code (1.67.0)
        \item 编译器:g++ (GCC) 11.2.0
        \item 操作系统:ArchLinux 5.17.5-zen1-1-zen (64-bit)
    \end{itemize}
\end{itemize}

\vspace{5pt}

\paragraph{解答} 时间复杂度:$\Theta(N \log N)$

源码:
\inputminted[bgcolor=codebg,frame=lines,autogobble,linenos=true,breaklines]{cpp}{src/t3.cpp}

\vspace{5pt}

\paragraph{实验结果}
(可附上截图)


\vspace{5pt}





\section*{三、心得体会}
    可根据“实验思考”部分作答,也可以根据个人具体体会作答。自己算法的创新点可在此处进行介绍,酌情加分。

\end{document} 
