\documentclass{ctexbeamer}
\usepackage{amsmath,amssymb}
\usepackage{amsthm}
\usepackage{subfig}
\usepackage{graphicx}
\usepackage[defaultmono,scale=0.85]{droidsansmono}
\usepackage{tikz}
\usepackage{minted}
\usepackage{color}
% \usepackage{mdframed}
% \usepackage[colorlink=true,xetex]{hyperref}

\usetheme{Madrid}
\usefonttheme[onlymath]{serif}
\definecolor{codebg}{rgb}{0.95,0.95,0.95}
\setCJKmonofont{WenQuanYi Micro Hei Mono}
% \surroundwithmdframed{minted}

\usetikzlibrary{graphdrawing, graphs, quotes, shapes}
\usegdlibrary{trees}
\tikzset{
    noleaf/.style={draw,ellipse,fill=red!60},
    ref/.style={draw,ellipse,fill=green!40},
    normal/.style={draw,ellipse},
    every picture/.style={}
}

\setminted{bgcolor=codebg,autogobble,breaklines}
\setmintedinline{bgcolor=codebg}

\newcommand{\theauthor}{Sparky\_14145}
\newcommand{\theinst}{College of Software Engineering}
\newcommand{\theshortinst}{CoSE}

% \input{personal_info/info.tex}

\author{\theauthor}
\institute[\theshortinst]{\theinst}
\title{线性规划与 LPBoost}

\begin{document}
    \begin{frame}
        \titlepage
    \end{frame}

    \begin{frame}
        \frametitle{目录}
    
        \tableofcontents
    
    \end{frame}

    \section{简介}
    \begin{frame}
        \frametitle{简介}
    
        LPBoost 的基本思想是,通过将一系列的弱分类器 $h_j : \pmb{\chi} \to \{-1, 1\}$ 进行线性组合,得到一个强分类器 $f(\pmb{x})$,即:

        \[
            f(\pmb{x}) = \sum_{i=1}^{n} \alpha_i h_i(\pmb{x})
        \] \pause

        并且尝试最大化得到的强分类器 $f$ 的最小边界(margin):

        \[
            \rho = \min_{k = 1,\ldots,m}\{\sum_{i=1}^n \alpha_i y_k h_i(\pmb{x}_k) \}
        \] \pause

        那么,该如何确定这里每个弱分类器的权重呢?\pause{}线性规划!
    
    \end{frame}

    \section{线性规划}
    \begin{frame}
        \frametitle{线性规划}
    
        线性规划是一类数学问题,标准形式如下:

        \begin{equation}
            \begin{aligned}
                \max\; & \pmb{c}^\mathrm{T}\pmb{x} \\
                \mathrm{s.t.}\; & \pmb{Ax} \leq \pmb{b} \\
                 & \pmb{x} \geq 0
            \end{aligned}
        \end{equation} \pause

        其中 $\pmb{c}$ 是一个系数向量,$\pmb{x}$ 是带求的最优解,$\pmb{Ax} \leq \pmb{b}$ 则是各个约束式。
    
    \end{frame}

    \begin{frame}
        \frametitle{求解线性规划}
        \framesubtitle{线性规划}
    
        求解线性规划的方法有很多:

        \begin{enumerate}
            \item 单纯形法;\pause
            \item 内点法;\pause
            \item 列构造法;\pause
            \item 等等... \pause
        \end{enumerate}

        其中,出于效率原因, LPBoost 使用的是列构造法。\pause
    
    \end{frame}

    \section{LPBoost}

    \begin{frame}
        \frametitle{LPBoost}
        \framesubtitle{软边界 LPBoost}
    
        LPBoost 的线性规划问题看起来有一点像软边界支持向量机:

        \begin{align*}
            \min_{\pmb{\alpha,\xi},\rho} & -\!\!\rho + D \sum_{i=1}^m \xi_i & \\
            \mathrm{s.t.} & \sum_{i=1}^n \alpha_i y_k h_i(\pmb{x}_k) + \xi_k \geq \rho, & k = 1,\ldots,m \\
             & \sum_{i=1}^n \alpha_i = 1 & \\
             & \xi_k \geq 0, \alpha_i \geq 0, \rho \in \mathbb{R} &
        \end{align*} \pause

        其中:$D$ 是需要在学习过程中手动调整的惩罚参数,$\pmb{\xi}$ 是松弛变量。
    
    \end{frame}

    \begin{frame}
    
        实践中一般选取 $D = \frac{1}{m\nu}$,其中 $\nu \in (\frac{1}{m}, 1)$,$m$ 为训练集样本总数。此时参数 $\nu$ 会拥有一些性质:\pause

        \begin{enumerate}
            \item $\nu$ 是训练集分类错误比例的上界,即设 $k$ 表示分类错误的样本,则 $\frac{k}{m} \leq \nu$;\pause
            \item $\nu$ 是分类边界上或边界外的训练样本比例的下界。
        \end{enumerate} \pause

        如果选择的 $\nu$ 得当,那么最优解所对应的权重向量 $\pmb{\alpha}$ 将会很稀疏,即只有一小部分弱分类器实际参与分类工作;且得到的模型只依赖训练样本中较小的子集,从而提高分类器的实际性能。
    
    \end{frame}

    \section{扩展 LPBoost}
    \subsection{硬边界}

    \begin{frame}
        \frametitle{扩展 LPBoost}
        \framesubtitle{硬边界 LPBoost}

        实际上如同支持向量机可以有软硬边界一般,LPBoost 也有软边界与硬边界之分。刚刚所介绍的 LPBoost 便是软边界,而硬边界 LPBoost 的线性规划问题如下:

        \begin{align*}
            \min_{\pmb{\alpha},\rho} & -\!\!\rho & \\
            \mathrm{s.t.} & \sum_{i=1}^n \alpha_i y_k h_i(\pmb{x}_k) \geq \rho, & k = 1,\ldots,m \\
             & \sum_{i=1}^n \alpha_i = 1, \alpha_i \geq 0, \rho \in \mathbb{R} &
        \end{align*} \pause
    \end{frame}

    \begin{frame}

        可以看出,硬边界 LPBoost 就是将软边界 LPBoost 的松弛变量去掉,使得约束条件强制要求所有的样本点都必须分类正确。\pause
        
        由于所有样本都分类正确,所以也不再需要向处于边界范围内的样本施加惩罚,故硬边界 LPBoost 的目标函数也就是简单的最小边界,即 $-\rho$。

    \end{frame}

    \subsection{LPBoost 优化回归}

    \begin{frame}
        \frametitle{LPBoost 优化回归}
        \framesubtitle{Confidence-rated Boosting}
    
        在之前的叙述中,不难发现,LPBoost 的线性规划问题(以及,其求解过程)中并没有要求数据的标签 $y_i \in \{+1, -1\}$,故我们可以考虑通过修改线性规划问题,来使得 LPBoost 可以处理回归问题。\pause
        
        实际上,不仅仅是最小边界,LPBoost 可以被用于优化任何能被变形成线性规划问题的代价函数。
    
    \end{frame}

\end{document}
