\documentclass{ctexrep}
\usepackage[chapter]{minted}
\usepackage{xcolor}
\usepackage{fontspec}
\usepackage{graphicx}
\usepackage{subfig}
\usepackage[hmargin=1.25in,vmargin=1in]{geometry}
\usepackage[colorlinks=true,bookmarks=true,bookmarksnumbered=true]{hyperref}

\setmonofont{WenQuanYi Micro Hei Mono}
\ctexset{
    chapter/name = {第,题}
}
\definecolor{mintbg}{rgb}{0.95,0.95,0.95}
\renewcommand{\listingscaption}{源代码}

\author{71120226 陈宇轩}
\title{数据结构作业六}

\begin{document}
    \maketitle

    \tableofcontents

    \chapter{P339: 2}

    \paragraph{(a)小问} 见表 \ref{table:p1_1}。

    \begin{table}[hp]
        \centering
        \caption{第 1 题:结点度数表}
        \label{table:p1_1}
        \begin{tabular}{|*{7}{c|}}
            \hline
            结点 & 0 & 1 & 2 & 3 & 4 & 5 \\ \hline
            入度 & 3 & 2 & 1 & 1 & 2 & 2 \\ \hline
            出度 & 0 & 2 & 2 & 3 & 1 & 3 \\ \hline
        \end{tabular}
    \end{table}

    \paragraph{(b)小问} 见表 \ref{table:p1_2}。

    \begin{table}[hp]
        \centering
        \caption{第 1 题:邻接矩阵}
        \label{table:p1_2}
        \begin{tabular}{|*{7}{c|}}
            \hline
              & 0 & 1 & 2 & 3 & 4 & 5 \\ \hline
            0 & 0 & 0 & 0 & 0 & 0 & 0 \\ \hline
            1 & 1 & 0 & 0 & 0 & 0 & 0 \\ \hline
            2 & 0 & 1 & 0 & 0 & 0 & 1 \\ \hline
            3 & 0 & 0 & 1 & 0 & 1 & 1 \\ \hline
            4 & 1 & 0 & 0 & 0 & 0 & 0 \\ \hline
            5 & 1 & 1 & 0 & 0 & 1 & 0 \\ \hline
        \end{tabular}
    \end{table}

    \paragraph{(c)小问} 见表 \ref{table:p1_3}。

    \begin{table}[hp]
        \centering
        \caption{第 1 题:邻接表}
        \label{table:p1_3}
        \subfloat[点表]{
            \begin{tabular}{|*{7}{c|}}
                \hline
                索引 & 0 & 1 & 2 & 3 & 4 & 5 \\ \hline
                \mintinline{text}{head} & 0 & 2 & 4 & 7 & 8 & 11 \\ \hline
            \end{tabular}
        }
        \\
        \subfloat[边表]{
            \begin{tabular}{|*{12}{c|}}
                \hline
                索引 & 1 & 2 & 3 & 4 & 5 & 6 & 7 & 8 & 9 & 10 & 11 \\ \hline
                \mintinline{text}{to} & 0 & 3 & 1 & 5 & 2 & 4 & 5 & 0 & 0 & 1 & 4 \\ \hline
                \mintinline{text}{next} & 0 & 1 & 0 & 3 & 0 & 5 & 6 & 0 & 0 & 9 & 10 \\ \hline
            \end{tabular}
        }
    \end{table}

    \paragraph{(d)小问} 见表 \ref{table:p1_4}。

    \begin{table}[hp]
        \centering
        \caption{第 1 题:邻接表}
        \label{table:p1_4}
        \subfloat[点表]{
            \begin{tabular}{|*{7}{c|}}
                \hline
                索引 & 0 & 1 & 2 & 3 & 4 & 5 \\ \hline
                \mintinline{text}{head_from} & 0 & 2 & 4 & 7 & 8 & 11 \\ \hline
                \mintinline{text}{head_to} & 9 & 10 & 5 & 2 & 11 & 7 \\ \hline
            \end{tabular}
        }
        \\
        \subfloat[边表]{
            \begin{tabular}{|*{12}{c|}}
                \hline
                索引 & 1 & 2 & 3 & 4 & 5 & 6 & 7 & 8 & 9 & 10 & 11 \\ \hline
                \mintinline{text}{from} & 1 & 1 & 2 & 2 & 3 & 3 & 3 & 4 & 5 & 5 & 5 \\ \hline
                \mintinline{text}{to} & 0 & 3 & 1 & 5 & 2 & 4 & 5 & 0 & 0 & 1 & 4 \\ \hline
                \mintinline{text}{next_from} & 0 & 1 & 0 & 3 & 0 & 5 & 6 & 0 & 0 & 9 & 10 \\ \hline
                \mintinline{text}{next_to} & 0 & 0 & 0 & 0 & 0 & 0 & 5 & 1 & 2 & 3 & 6 \\ \hline
            \end{tabular}
        }
    \end{table}

    \paragraph{(e)小问} 共有 3 组强连通分量:\{0\},\{1, 2, 3, 5\} 与 \{4\}。

    \chapter{P372: 1}

    \paragraph{思路} 从树根开始向叶子搜索,每搜索到一个子结点就将其距离设为亲代结点到根的距离加上边长。
    最终可以计算出每个结点到根的距离。原因:树上任意两点之间有且只有一条路径(不允许反复经过一条边),
    而反复经过一条边不会令路径长度减小,因此从根 DFS 到结点的路径必为最短路。

    \paragraph{代码} 见源代码 \ref{listing:p2}。

    \begin{listing}[hp]
        \centering
        \caption{求树上路径长度}
        \label{listing:p2}
        \begin{minted}[autogobble,linenos=true,breaklines=true,bgcolor=mintbg]{cpp}
            /**
             * @brief 求树根到其他所有点的路径长度。
             * 结果存在 v[x].dist 中,v 是保存结点信息的数组,x 是对应结点编号。
             * 假设图以邻接单表形式存储在 v(点表) 和 e(边表)中。
             *
             * @param p 当前访问的结点。
             * @param f 当前结点 p 的亲代结点。对于树根,f 可取任意不在树中的结点编号。
             */
            void get_dist(int p, int f)
            {
                for (int i = v[p].head; i; i = e[i].next)
                {
                    // 当前边 i 的终点
                    int cur = e[i].to;

                    if (cur == f)
                    {
                        continue;
                    }

                    v[cur].dist = v[p].dist + e[i].weight;
                    get_dist(cur, p);
                }
            }
        \end{minted}
    \end{listing}

    \chapter{P389: 2}

    \paragraph{(a)小问} 见表 \ref{table:p3_cal}

    \begin{table}[hp]
        \centering
        \caption{第 3 题计算表}
        \label{table:p3_cal}
        \begin{tabular}{|*{11}{c|}}
            \hline
            $x$    & 0 & 1 & 2 & 3  & 4  & 5  & 6  & 7  & 8  & 9 \\ \hline
            $e(x)$ & 0 & 5 & 6 & 12 & 15 & 16 & 16 & 19 & 21 & 23 \\ \hline
            $l(x)$ & 0 & 9 & 6 & 12 & 15 & 19 & 16 & 19 & 21 & 23 \\ \hline
        \end{tabular}
    \end{table}

    \paragraph{(b)小问} 最短用时为 23。

    一条最长路:\{ $0 \rightarrow 2 \rightarrow 3 \rightarrow 6 \rightarrow 8 \rightarrow 9$ \}。

    \paragraph{(c)小问} Critical activities: 0, 2, 3, 4, 6, 7, 8 和 9。

    \paragraph{(d)小问} 0,2,3,8 四个活动的提速都能提速整个任务。
\end{document}